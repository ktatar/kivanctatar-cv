%%%%%%%%%%%%%%%%%%%%%%%%%%%%%%%%%%%%%%%%%
% "ModernCV" CV and Cover Letter
% LaTeX Template
% Version 1.1 (9/12/12)
%
% This template has been downloaded from:
% http://www.LaTeXTemplates.com
%
% Original author:
% Xavier Danaux (xdanaux@gmail.com)
%
% License:
% CC BY-NC-SA 3.0 (http://creativecommons.org/licenses/by-nc-sa/3.0/)
%
% Important note:
% This template requires the moderncv.cls and .sty files to be in the same 
% directory as this .tex file. These files provide the resume style and themes 
% used for structuring the document.
%
%%%%%%%%%%%%%%%%%%%%%%%%%%%%%%%%%%%%%%%%%

%----------------------------------------------------------------------------------------
%	PACKAGES AND OTHER DOCUMENT CONFIGURATIONS
%----------------------------------------------------------------------------------------

\documentclass[10pt,letterpaper]{moderncv} % Font sizes: 10, 11, or 12; paper sizes: a4paper, letterpaper, a5paper, legalpaper, executivepaper or landscape; font families: sans or roman

\moderncvstyle{classic} % CV theme - options include: 'casual' (default), 'classic', 'oldstyle' and 'banking'
\moderncvcolor{grey} % CV color - options include: 'blue' (default), 'orange', 'green', 'red', 'purple', 'grey' and 'black'
\usepackage{lipsum}
\usepackage{amsmath}
\usepackage{url}

\usepackage[scaled]{beramono}
\renewcommand*\familydefault{\ttdefault} %% Only if the base font of the document is to be typewriter style
\usepackage[T1]{fontenc}

\usepackage[scale=0.87]{geometry} % Reduce document margins
\linespread{0.9}
\setlength{\hintscolumnwidth}{2.16cm} % Uncomment to change the width of the dates column
%\setlength{\makecvtitlenamewidth}{10cm} % For the 'classic' style, uncomment to adjust the width of the space allocated to your name

%----------------------------------------------------------------------------------------
%	NAME AND CONTACT INFORMATION SECTION
%----------------------------------------------------------------------------------------

\firstname{K{\i}van\c{c}} % Your first name
\familyname{Tatar} % Your last name

% All information in this block is optional, comment out any lines you don't need
\title{Curriculum Vitae}
%\address{Chalmers University of Technology \\Interaction Design Unit \\ Computer Science Department \\ Kuggen, Lindholmsplatsen 1\\}{ 417 56 Gothenburg, Sweden}
%\mobile{(778) 858 6073}
%\phone{(000) 111 1112}
%\fax{(000) 111 1113}
\email{info@kivanctatar.com}
\homepage{kivanctatar.com}{Personal Website} % The first argument is the url for the clickable link, the second argument is the url displayed in the template - this allows special characters to be displayed such as the tilde in this example
%\extrainfo{\href{http://metacreation.net/}{Metacreation Lab} \\ \href{http://musicalmetacreation.org}{Musical Metacreation}}

%\photo[79pt][0.8pt]{picture} % The first bracket is the picture height, the second is the thickness of the frame around the picture (0pt for no frame)
%\quote{"A witty and playful quotation" - John Smith}

%----------------------------------------------------------------------------------------
\begin{document}
	%\ttfamily
	%\asciifamily
	\makecvtitle % Print the CV title
	
	%\cventry{\textcolor{magenta}{Date of Birth}}{31.08.1988}
	%\cventry{\textcolor{magenta}{Nationality}}{The Republic of Turkey}
	
	%----------------------------------------------------------------------------------------
	%	EDUCATION SECTION
	%----------------------------------------------------------------------------------------
	%\cvitem{ \textcolor{magenta}{Date of Birth}}{31.08.1988}
	%\cvitem{ \textcolor{magenta}{Nationality}}{The Republic of Turkey}
		
%	\section{Nationality}
%	\cvitem{Turkey}{Citizen, by born}
%	\cvitem{Canada}{Permanent Resident, since 2020-07-07}
	
	\section{Education}
	
	\cvitem{07/2019 09/2014}{Ph.D., Interactive Arts and Technology (SIAT), Simon Fraser University (SFU), \textit{Greater Vancouver, BC Canada}}
	
	\section{Artworks}
	
	\cvitem{2021-12}{\href{https://kivanctatar.com/Plastic-Biosphere}{\textbf{Plastic Biosphere No.2}}, a live performance stream, using AI for music and moving images, as a part of \href{https://www.sacmmt.com/}{Bowed Electrons 2021} hosted from \textcolor{magenta}{Cape Town, South Africa}, invited through curatorial selection.}
	
	\cvitem{2021-09-23}{\href{https://kivanctatar.com/Plastic-Biosphere}{\textbf{Plastic Biosphere No.2}} as a part of collective concert \href{http://mutablesubject.ca/interplay_2021/}{interplay\_2021} with Sidi Chen, Find Mutya, Brandy Leary, Tamar Tabori, Raj Gill, Stéphanie Cyr, hosted from Vancouver, BC, Canada.}
	
	\cvitem{2021-08-16}{\href{https://kivanctatar.com/Plastic-Biosphere}{\textbf{Plastic Biosphere No.2}} premier with \href{https://www.instagram.com/tamamamar}{Tamar Tabori}, a live performance stream using AI for music and moving images, with the support of \href{https://newmusic.org/}{Vancouver New Music}.}
	
	\cvitem{2021-07-15 2021-04-15}{\href{https://kivanctatar.com/gestalt-generation}{\textbf{gestalt generation no.1}} with \href{https://remysiu.com/}{Remy Siu}, as a part of collective exhibition \href{https://event.culture.tw/NTMOFA/portal/Registration/C0103MAction?useLanguage=en&actId=10050&request_locale=en}{\textit{Aethereal}}, at the \href{https://www.ntmofa.gov.tw/en/}{Taiwan National Museum of Fine Arts}, \textcolor{magenta}{Taichung, Taiwan}.}
	
	\cvitem{2021-01-03 2020-12-07}{\href{https://kivanctatar.com/Brush-of-AI}{\textbf{The Silhouettes of Istanbul No.3: The Brush of Artificial Intelligence}}, a video work showcased on 52 public large-scale screens scattered around Istanbul, as a part of Istanbul The Lights festival, produced by the \href{https://ciacef.org/}{Contemporary Istanbul Foundation (CIF)} and the \textcolor{magenta}{City of Istanbul, Turkey}, three NFTs from this series is available on \href{https://foundation.app/@kivanctatar}{Foundation}, invited through CIF curatorial selection.}
	
	\cvitem{2020-09-09}{\textbf{Plastic Biosphere No.1} an interactive artwork using AI for music and moving images, exhibited as a part of the Kepler's Gardens Series, \textcolor{magenta}{Bucharest, Romania} Edition, at the \textcolor{magenta}{Ars Electronica 2020} in \textcolor{magenta}{Linz, Austria}.}
	
	\cvitem{2020-08-30}{\href{https://www.youtube.com/watch?v=tC9Cr1YDtHo}{\textbf{Instance}} telematic music-dance performance of Lucy Strauss and Sumalgy Nuro with the contributions of K{\i}van\c{c} Tatar, Bob Pritchard, Marina Thibeault, presented as online live stream.}
	
	\cvitem{2020-06 2019-12}{\href{https://kivanctatar.com/Silhouettes-of-Istanbul}{\textbf{The Silhouettes of Istanbul}} In the collective exhibition \textit{Genetic Codes of Turkish Design, 4th Edition}, with ... Refik Anadol, ... at the International Departures Lobby, \textcolor{magenta}{Istanbul Airport, Turkey}, invited through curatorial selection.}
	
	%\cvitem{2019-11-30}{\textbf{\href{https://kivanctatar.com/revive}{REVIVE}: \href{http://bocadellupo.com/revive-a-coda-guest-presentation/}{A CODA Guest Presentation}} [Performer \& Developer] concert with Philippe Pasquier and \href{remysiu.com}{Remy Siu} at the \href{http://bocadellupo.com/revive-a-coda-guest-presentation/}{Performance Works} (14 channel, 2 projections), co-produced by \href{http://bocadellupo.com/revive-a-coda-guest-presentation/}{Boca Del Lupo} Vancouver, BC, Canada.}
	
	\cvitem{2019-10}{\href{https://kivanctatar.com/Machine-Learned-Landscape-01}{\textbf{Landscapes Past Futures}} [Artist \& Technologist] Prints and fixed-media video, in the exhibition as a part of the \textit{Index Media Arts Festival}, at the \textit{gnration} Gallery,  \textcolor{magenta}{Braga, Portugal}.}
	
	\cvitem{2019-07-05}{\href{https://kivanctatar.com/u1m}{\textbf{Under 1 Minute (U1M):  Seismic Waves}} [Artist \& Technologist] concert at the \href{http://www.starofbosphorus.com.tr}{Star Bosphorus Data Center} with Mehmet \"Unal and Hakan Y{\i}lmaz, \textcolor{magenta}{\.Istanbul, Turkey}, invited through curatorial selection}
	
	\cvitem{2019-09-12}{ \href{https://kivanctatar.com/u1m}{\textbf{Under 1 Minute (U1M): Digital Sculpture "The One"}} [Artist \& Technologist] exhibition at the \href{http://www.starofbosphorus.com.tr}{Contemporary Istanbul} PlugIn Exhibition, \textcolor{magenta}{74000 visitors, 40k shares and 10 million views on social media within a week}, with Mehmet \"Unal and Hakan Y{\i}lmaz, \textcolor{magenta}{\.Istanbul, Turkey}, invited through CIF curatorial selection}
	
	\cvitem{2019-07-18}{\href{https://kivanctatar.com/u1m}{\textbf{Under 1 Minute}} (U1M) [Artist \& Technologist] concert at the Harbiye Cemil Topuzlu Open-Air Theatre, the concert opening performance for Kenan Do\u{g}ulu (Turkish pop-star), approx. \textcolor{magenta}{10000 audience}, with Mehmet \"Unal and Hakan Y{\i}lmaz, \textcolor{magenta}{\.Istanbul, Turkey}}
	
	\cvitem{2019-06-20}{\href{http://kivanctatar.com/revive}{\textbf{REVIVE}} [Performer \& Technologist] concert at \href{https://nycemf.org/}{New York Electroacoustic Music Festival (NYCEMF 2019)} with Philippe Pasquier and \href{remysiu.com}{Remy Siu} at the \href{https://www.fridmangallery.com/}{Fridman Gallery} (8 channel), \textcolor{magenta}{New York City, New York, USA}.}
	
	\cvitem{2019-04-03}{\href{https://kivanctatar.com/u1m}{\textbf{Under 1 Minute}} [Artist \& Technologist] concert at the \href{http://ameistanbul.com/events/ace-of-m-i-c-e-awards-ceremony/}{Ace of M.I.C.E Awards Ceremony}{ with Mehmet \"Unal and Hakan Y{\i}lmaz}, \textcolor{magenta}{\.Istanbul, Turkey}, invited through curatorial selection.}
	
	\cvitem{2019-02-09}{\href{https://kivanctatar.com/revive}{\textbf{REVIVE}} [Performer \& Technologist] concert at the \href{https://livingthingsfestival.com/}{Living Things Festival 2019} with Philippe Pasquier, at Kelowna Art Gallery (6 speakers), \textcolor{magenta}{Kelowna, BC, Canada}, invited through curatorial selection.}
	
	\cvitem{2018-12-14}{\href{https://kivanctatar.com/zeta}{\textbf{ZETA}} [Artist \& Technologist] touch-based interactive installation for 360 immersive interface with multichannel audio, at the \href{http://immersivelab.zhdk.ch/}{Immersive Lab} at \href{https://www.zhdk.ch/en/research/icst}{the Institute for Computer Music and Sound Technologies}, \href{https://www.zhdk.ch/}{Zurich University of the Arts} with \href{https://philipepasquier.com}{Philippe Pasquier}; \textcolor{magenta}{Z\"urich, Switzerland}, invited through curatorial selection, supported by the grant (\hyperref[CCA2]{CCA-2}) that is acquired competitively.}
	
	\cvitem{2018-12-08}{\href{https://kivanctatar.com/revive}{\textbf{REVIVE}} [Performer \& Technologist] concert at \href{https://www.zhdk.ch/veranstaltung/37286}{Zurich University of the Arts} with Philippe Pasquier and \href{http://remysiu.com}{Remy Siu} at the Konzertsaal 2 (26 speakers and 4 subwoofers), \textcolor{magenta}{Z\"urich, Switzerland}}
	
	\cvitem{2018-08-23}{\href{https://kivanctatar.com/revive}{\textbf{REVIVE}} [Performer \& Technologist] concert at \href{https://www.mutek.org/en/archives/events/2018/1531-satosphere-3-revive}{\textcolor{magenta}{MUTEK Montreal 2018}} with Philippe Pasquier and \href{http://remysiu.com}{Remy Siu} at the Soci\`et\`e des Arts Technologique (SAT) dome (157 speakers and 8 projectors), 2 performances, \textcolor{magenta}{Montreal, Quebec, Canada}, invited through curatorial selection, supported by the grant (\hyperref[CCA2]{CCA-2}) that is acquired competitively.}
	
	\cvitem{2018-08-31}{\href{http://kivanctatar.com/respire}{\textbf{RESPIRE}} [Artist \& Technologist] exhibition at \href{http://cinevolutionmedia.com/portfolio-item/dc2018-air/}{Digital Carnival 2018} with Mirjana Prpa and Philippe Pasquier Vancouver, BC, Canada, invited through curatorial selection.}
	
	\cvitem{2018-05-20}{\textbf{In the Engine} [Composer] in the collective concert, \href{https://www.facebook.com/events/s/istanbul-soundscape-project-ha/139446233576758/?ti=icl}{\.{I}stanbul Soundspace Project: Haydarpa\c{s}a'da Bir Gar.} @\href{http://www.arkaoda.com/}{arkaoda}, \textcolor{magenta}{\.{I}stanbul, Turkey}, invited through curatorial selection.}
	
	\cvitem{2018-04-24}{\href{https://kivanctatar.com/revive}{\textbf{REVIVE}} [Performer\&Developer] concert at \href{http://sat.qc.ca/fr/connexions}{CHI Connexitions} with Philippe Pasquier and \href{http://remysiu.com}{Remy Siu} at the Soci\`et\`e des Arts Technologique (SAT) dome (157 speakers and 8 projectors), 7 performances as a part of \href{https://chi2018.acm.org/}{\textcolor{magenta}{CHI 2018}} Conference on Human Factors in Computing Systems, \textcolor{magenta}{Montreal, Quebec, Canada}}
	
	\cvitem{2018-04-27}{\href{http://kivanctatar.com/respire}{\textbf{RESPIRE}} [Artist \& Technologist] exhibition at \href{https://chi2018.acm.org/}{CHI Virtual Reality exhibiton} with Mirjana Prpa and Philippe Pasquier, as a part of \href{https://chi2018.acm.org/}{\textcolor{magenta}{CHI 2018}} Conference on Human Factors in Computing Systems, \textcolor{magenta}{Montreal, Quebec, Canada}.}
	
	\cvitem{2018-04-18}{\href{http://kivanctatar.com/respire}{\textbf{RESPIRE}} [Artist \& Technologist] exhibition at \href{http://vanartgallery.bc.ca/}{\textcolor{magenta}{Vancouver Art Gallery}} with  Mirjana Prpa and Philippe Pasquier as a part of \href{http://mwx2018.org/}{the conference Museums and the Web MWX18}, Vancouver, BC, Canada}
	
	\cvitem{2018-03-08}{\textbf{Eternal Pink Noise Machine} [Artist \& Technologist], sound installation with Philippe Pasquier, at the \textit{\href{https://www.facebook.com/events/217009415527752}{Pink Noise Pop Up}} Exhibition in \textcolor{magenta}{Seoul, South Korea.}}
	
	\cvitem{2018-02-04}{\textbf{Trumpet \& Electronics} Solo performance at the collective concert \href{https://www.facebook.com/events/2032925513589565/}{\textit{Blue prints\_new prints}} with Alanna Ho, Ben Brown, and Roxanne Nesbitt; at the Gold Saucer Studio, Vancouver, Canada}
	
	% \cvitem{2018::2017}{\textbf{Collaborator \& Developer} \href{http://lindabouchard.com/}{Linda Bouchard}'s Live Structures project}
	
	\cvitem{2017-11-16}{\href{https://www.nowsociety.org/event/trading-places-un\%C3\%A9change-dimprovisateurs-montr\%C3\%A9al-vancouver}{\textbf{Trading Places: Un \'{E}change d'Improvisateurs}} Concert [Trumpet\&Electronics]; with Vicky Mettler, Torsten Muller, and Ross Birdwise; at the Roundhouse, Vancouver, Canada}
	
	\cvitem{2017-09-24}{\textbf{Theta} [Artist \& Technologist] MASOM joins two media art companies from Istanbul, \href{http://ouchhh.tv/}{Ouchhh} and \href{http://audiofil.io/}{AudioFil} for a projection mapping piece on \textcolor{magenta}{the Facade of the Bolshoi Theatre}, at the \href{http://lightfest.ru/en/}{Circle of Light 2017}, \textcolor{magenta}{Moscow, Russia}}
	
	\cvitem{2017-09-16}{\textbf{Theta} [Artist \& Technologist] MASOM joins two media art companies from Istanbul, \href{http://ouchhh.tv/}{Ouchhh} and \href{http://audiofil.io/}{AudioFil} for \href{http://www.imapp.ro/2017-2/}{a projection mapping piece} on \textcolor{magenta}{the Facade of the Romanian Parliament}, at the \href{http://imapp.ro/}{IMapp Bucharest 2017}, \textcolor{magenta}{Bucharest, Romania}}
	
	\cvitem{2017-09-07}{\href{https://www.facebook.com/Ouchhh.tv/videos/1492873787455600/}{\textbf{IOTA\_AI}} [Artist \& Technologist], MASOM joins two media art companies from Istanbul, \href{http://ouchhh.tv/}{Ouchhh} and \href{http://audiofil.io/}{AudioFil} for a performance at the \href{https://www.aec.at/ai/en/iota}{\textcolor{magenta}{Ars Electronica Festival 2017}} with the theme Artificial Intelligence. The team performed three times at the Deep Space 8K during the festival. \textcolor{magenta}{Linz, Austria}}
	
	\cvitem{2017-06-19}{\textbf{MA\_Test SOM\_Pattern} [Performer \& Technologist] with the project \href{https://kivanctatar.wordpress.com/patar}{Patar}, in the collective concert by \href{https://musicalmetacreation.org/mume-2017-concert}{\textit{Musical Metacreation Concert}} \textcolor{magenta}{Atlanta, Georgia, USA}}
	
	%\cvitem{2017-04-22}{\textbf{Patar @CoCreaTive} [Performer \& Technologist] in the collective concert \textit{Barely Constrained} by \href{https://cocreative.wordpress.com/}{Co.Crea.Tive} Vancouver, BC, Canada}
	
	%\cvitem{2017-03-23}{\href{https://cocreative.wordpress.com/2017/04/30/a-big-masom-family/}{\textbf{A Big MASOM Family}} [Performer \& Technologist] in the collective concert \textit{RE-UN-SOLVED} by \href{https://cocreative.wordpress.com/}{Co.Crea.Tive} Vancouver, BC, Canada}
	
	%\cvitem{2016-12-25}{\textbf{Tatar and MASOM take the AID train} [Performer \& Technologist] in the collective concert \href{https://www.facebook.com/events/1769695369957515/}{\textit{Take the AID Train}} by \href{http://artisdead.in}{A.I.D}, \textcolor{magenta}{\.{I}stanbul, Turkey}}
	
	\cvitem{2016-12-02}{\href{https://www.nowsociety.org/madmethod-december-2-3}{\textbf{madMethod}} [Performer \& Technologist] by NOW Society, with Stefan Smulovitz, Sammy Chien, MASOM, Philippe Pasquier, Lisa Cay Miller, Jon Bentley, JP Carter, James Meger, Skye Brooks, at Orpheum Annex, Vancouver, BC, Canada}
	
	\cvitem{2016-11-10} {\href{https://kivanctatar.com/Pulse-Breath-Water}{\textbf{Pulse.Breath.Water}} [Artist \& Technologist] with Mirjana Prpa, Philippe Pasquier, Bernhard Reicke in the VR exhibition at \textcolor{magenta}{\href{http://www.mutek.org/en/img/2016/artworks}{MUTEK\_IMG}} - Virtual Reality (head mounted display and headphones), generative audio, embodied interaction (via breath sensors), \textcolor{magenta}{Montreal, Quebec, Canada}}
	
	%\cvitem{2016-10-22}{\href{https://kivanctatar.com/MASOM-0-01}{\textbf{A Conversation with AI}} [Performer \& Technologist] in the collective concert \textit{Open to Enter} by \href{https://cocreative.wordpress.com/}{CoCreaTive}, Vancouver, BC, Canada}
	
	%\cvitem{2016-08-31}{\href{https://smc2016.hfmt-hamburg.de/?session=musebot-chill-out-session-a-continuously-running-installation}{\textbf{Musebot Chill-out Session}} [Artist \& Technologist] with Arne Eigenfeldt, Paul Paroczai, Oliver Bown, Ben Carey, Toby Gifford, Jeffrey Morris, and Si Wait, Sound and Music Computing, SMC 2016, \textcolor{magenta}{Hamburg, Germany}}
	
	\cvitem{2016-07-18} {\textbf{\href{https://kivanctatar.com/POEMA}{P.O.E.M.A.}} [Artist \& Technologist] with Regina Miranda, Mirjana Prpa, Philippe Pasquier, and Bernhard Reicke; Generative Audio (quadrophonic setup), Choreographic Installation, Virtual Reality (head mounted display and projection), Embodied Interaction (via respiration sensors), at the gallery \textit{Oi Futuro}, as a part of the cultural program at \textcolor{magenta}{OLYMPICS 2016, Rio de Janeiro, Brazil}}
	
	%\cvitem{2016-07-11}{\textbf{Musebot Chill-out Session} [Artist \& Technologist] sound installation with Arne Eigenfeldt, Paul Paroczai, Oliver Bown, Ben Carey, Toby Gifford, and Jeffrey Morris, International Conference on New Interfaces for Musical Expression, NIME 2016, \textcolor{magenta}{Brisbane, Australia}}
	
	%\cvitem{2016-04-07} {\href{https://kivanctatar.com/Organic-Strategies}{\textbf{Organic Strategies}} [Trumpet \& Electronics] Matthew Ariaratnam in the collective concert: \textit{Constrained Improv}, @\textit{Red Gate Arts Society}, Vancouver, BC, Canada}
	
	%\cvitem{2016-03-10} {\href{https://kivanctatar.com/Pulse-Breath-Water}{\textbf{Pulse.Breath.Water}} [Artist \& Technologist] with Mirjana Prpa, Philippe Pasquier, and Bernhard Reicke, in the exhibition \href{http://oneartspace.com/2016/03/10/scorestraces-exposing-the-body-through-computation/}{\textit{Scores+Traces: exposing the body through computation}} - Virtual Reality (head mounted display and headphones), generative audio, embodied interaction (via breath sensors) @\textit{One Art Space}, \textcolor{magenta}{New York, NY, USA}}
	
	\cvitem{2016-03-10} {\textbf{Tuned Ocean no.2} [Artist \& Technologist] sound installation in the exhibition \textit{Scores+Traces: exposing the body through computation}  - sound installation, generative audio, @\textit{One Art Space}, \textcolor{magenta}{New York, NY, USA}.}
	
	%\cvitem{2016-01-23}{\href{https://kivanctatar.com/Code-of-Silence-2}{\textbf{Code of Silence Nb.2}} [Composer] graphic notation for any number of performers. Premiered by Plastic Acid Orchestra at One-Page Score event. Vancouver, BC, Canada}
	
	%\cvitem{2016-01-16}{\href{https://kivanctatar.com/Code-of-Silence}{\textbf{Code of Silence}} [Trumpet \& Electronics] in the collective concert - \textit{Improvised Resolutions} @\textit{Gold Saucer Studio}, Vancouver, BC, Canada}
	
	%\cvitem{2015-12-09}{\textbf{Musebot Chill-out Session} [Artist \& Technologist] with Arne Eigenfeldt, Paul Paroczai, Oliver Bown, Ben Carey, Toby Gifford, and Jeffrey Morris, Generative Art Conference 2016, \textcolor{magenta}{Venice, Italy}}
	
	%\cvitem{2015-11-16}{\href{https://kivanctatar.com/Musebots-for-PROCJAM-2015}{\textbf{Musebots for PROCJAM 2015}} [Artist \& Technologist] generative music piece with Arne Eigenfeldt, Oliver Bown, Ben Carey, Toby Gifford}
	
% 	\cvitem{2015-06-07}{\href{https://threelittlereddots.org/performances/together-apart/}{\textbf{Together () Apart}} [Sound Designer] performance piece by \href{https://threelittlereddots.org/}{Isabelle Kirouac}, Vancouver, BC, Canada}
	
% 	\cvitem{2015-08}{\href{https://kivanctatar.com/Black-and-white}{\textbf{Black and White: Where the bomb meets the toys}} [Composer] graphic notation for three performers, Vancouver, BC, Canada}
	
% 	\cvitem{2015-08-22}{\href{https://vinesartfestival.com/wp-content/uploads/2018/05/2015_festival_poster.jpg}{\textbf{Dissonant Disco Collective}} performance with made instruments and trumpet at \textit{Vines Festival}, Vancouver, BC, Canada}
	
% 	\cvitem{2015-04-23}{\href{https://www.facebook.com/events/sfu-school-for-the-contemporary-arts/antinomial-antiphonies-part-of-the-mixed-greens-performance-series/468911126593214/}{\textbf{Antiphons}} [Performer] composition by Ben Wylie \textit{Antinomial Antiphonies, Mixed Greens Performance Series}, SFU Woodwards, Vancouver, BC, Canada}
	
% 	\cvitem{2015-03-31}{\href{https://kivanctatar.com/Deep-Breath}{\textbf{Deep Breath}} [Trumpet \& Electronics] solo live performance, Black Box, Interactive Arts and Technology, SFU, Surrey, BC Canada}
	
% 	\cvitem{2014-05-19}{\href{https://kivanctatar.com/Tat-Kal-Dem}{\textbf{Tat-Kal-Dem trio}} [Trumpet \& Electronics] Karakedi, \.{I}stanbul, Turkey}
	
% 	\cvitem{2014-05-05}{\href{https://kivanctatar.com/Sonic-Arts-Day}{\textbf{Soundscapes from Poland}} [Trumpet] Sonic Arts Day Concert, free improvisation session, Mustafa Kemal Hall, \.{I}stanbul}
	
% 	\cvitem{2013-05-02}{\textbf{Tuned Ocean} for electronics and recorded piano, live performance, \href{https://www.facebook.com/events/140813849439513}{ELECTROSONIC CITY 3.0}, Borusan Music House, \.{I}stanbul}
	
% 	\cvitem{2013-03-16}{\textbf{Tin Men and the Telephone (NL) and Furt(DE, UK):} Do\u{g}a\c{c}lamada Avangard Perspektifler [Guest Musician, Trumpet \& Electronics]}
	
% 	\cvitem{2013-02-09}{\textbf{Take it} [Electronics] MIAM Groove Collective Concert, Wake Up Call, \.{I}stanbul}
	
% 	\cvitem{2012-12-05}{\href{https://www.facebook.com/events/565785350101463/}{\textbf{ Beyond Trumpet}} [Trumpet \& Electronics] live solo performance, four channels, MIAM NOISE COLLECTIVE XV Concert, MIAM Recording Studio, \.{I}stanbul}
	
% 	\cvitem{2012-10-31}{\textbf{\.{I}stanbul in Boring Stereo without Clarinet} [Composer] four channels electroacoustic piece, MIAM Electroacoustic Collective XV Concert, MIAM Recording Studio, \.{I}stanbul}
	
	%\cvitem{2012}{\textbf{A Jazz Installation} [Composer] performance with Sound Installation, METU Architecture Department Building, Ankara}
	
	%\cvitem{2011}{\textbf{Huzur} [Composer] a sound installation with participatory interaction, METU Library Exhibition Hall, ANKARA}
	
% 	\cvitem{2008-2012}{\textbf{Principal Trumpet}, METU Big Band, two concerts per year, Ankara}
	
	%\cvitem{2010}{\textbf{Bir Anar\c{s}istin Kaza Sonucu \"{O}l\"{u}m\"{u}} [Composer \& Music Producer], play by Dario Fo, Ayakba\u{g}{\i} Theatre, Ankara}
	
	%\cvitem{2010}{\textbf{Kanl{\i} Nigar} [Trumpet] Turkish Government Theatres, three performances, Ankara}
	
	%\cvitem{2009}{\textbf{\"{O}denmeyecek \"{O}demiyoruz} [Trumpet], a play Ayakba\u{g}{\i} Theatre, multiple performances, Ankara}
	
	\cvitem{2009-09-12}{\textbf{Point, Line, Space and Sound} [Trumpet \& Electronics] performance as a part of Bauhaus project, in collaboration between Bauhaus University and Middle East Technical University \textcolor{magenta}{Weimar, Germany}}
	
	%\cvitem{2009-07-30}{\textbf{Point, Line, Space and Sound} [Trumpet \& Electronics] performance as a part of Bauhaus project, in collaboration between Bauhaus University and Middle East Technical University, METU, Ankara, Turkey}
	
	%\cvitem{2008}{\textbf{Hayat{\i}m{\i}z Bi M\"{u}zika} [\texttt{Trumpet}] musical play \.{I}zmir, Turkey}
	
	%\cvitem{2008}{\textbf{Into the Woods} [Trumpet] musical play, METU, Ankara, Turkey}
	
	%\cvitem{Pre-2008}{\textbf{Trumpet}, several bands playing genres such as mainstream, jazz, western classical, funk, reggae, swing, avangard}
	
	\section{Press}
	\cvitem{2020-05-22}{\textbf{Neural} - \href{http://neural.it/2020/05/respire-breathing-in-sound-and-vision/}{\textit{Respire, breathing in sound and vision}}}
	
	\cvitem{2019-12-12}{\textbf{Exclaim!} - \href{https://exclaim.ca/music/article/the_artificial_intelligence_takeover}{\textit{The Artificial Intelligence Takeover of Music in 2019}}, with Holly Herndon, YACHT, Endel, and Algorave artists.}
	\cvitem{2019-12-10}{\textbf{The La Source}, Volume 20, Issue 09 -  \href{http://thelasource.com/en/2019/11/18/kivanc-tatar-crossing-the-boundaries-of-science-and-the-arts/}{\textit{Kıvanç Tatar: crossing the boundaries of science and the arts}}, authored by Xi Chen.}
	\cvitem{2019-10-16}{\textbf{SFU News} - \href{https://www.sfu.ca/sfunews/stories/2019/10/graduate-takes-new-media-to-new-creative-levels-with-artificial-.html?fbclid=IwAR3gZqtUby9aNQUr0-AiWr_47nLJ1tXY3LGFCLIbE297u-OQYArU0o5qM78}{\textit{Graduate takes New Media to new creative levels with Artificial Intelligence}}, Vancouver, BC, Canada}
	
	\cvitem{2020-06-19}{\textbf{SFU News} - \href{https://www.sfu.ca/siat/stories/research/exploring-creative-artificial-intelligence.html}{\textit{Exploring Creative Artificial Intelligence}}, Vancouver, BC, Canada}
	
	\cvitem{2019-01-01}{\href{https://kivanctatar.com/2019-istanbul-art-news}{\textbf{Istanbul Arts News}} - Piyasa - January Issue, Interview on Creative AI, Turkey, authored by G\"uniz An\i l}
	
	\cvitem{2018-01-04}{\textbf{Artful Living}, \href{https://www.artfulliving.com.tr/sanat/dev-kadro-iki-gorsel-sanatci-bir-besteci-ve-bir-yapay-zek-i-14437}{\textit{Dev Kadro: \.Iki G\"orsel Sanat\c{c}{\i}, Bir Besteci ve Bir Yapay Zeka}}, authored by Esra \"Ozkan, Turkey}
	
	\cvitem{2016-11-06}{\textbf{VANDOCUMENT}, \href{https://vandocument.com/2016/11/a-conversation-with-artificial-intelligence/}{\textit{A Conversation with Artificial Intelligence}}, authored by Ash Tanasiychuk, Vancouver, BC, Canada}
	
	\cvitem{2016-08-08}{\textbf{MetroNews Vancouver}, local newspaper front page, \textit{Vancouver artists making waves at the Olympics}, Vancouver, BC, Canada}
	
	\cvitem{2016-08-09}{\textbf{Daily Hive}, \href{http://dailyhive.com/vancouver/rio-olympics-sfu-art-installation}{\textit{Rio Olympics showcases SFU virtual reality dance installation}}, Vancouver, BC, Canada}
	
	\cvitem{2016-08-17}{\href{https://kivanctatar.com/POEMA-SFU-news}{\textbf{SFU News}, \textit{SIAT art project at Rio Olympics takes your breath away}}, authored by Allen Tung, Vancouver, BC, Canada}
	
	\cvitem{2016-07-24}{\textbf{O Imparcial}, \href{https://oimparcial.com.br/entretenimento-e-cultura/2016/07/performance-imersiva-onde-o-publico-experimenta-a-realidade-virtual/}{\textit{Imersiva onde o publico experimenta a realidade virtual}, authored by Camila Pereira}, \textcolor{magenta}{Brazil}}
	
	\cvitem{24/07/2016}{\href{https://www.acritica.com/channels/entretenimento/news/companhia-de-regina-miranda-mistura-danca-e-realidade-virtual-em-obra-que-esta-em-cartaz-no-rj}{\textbf{A Critica}, \textit{Companhia de Regina Miranda mistura danca e realidade virtual em obra que esta em cartaz no rj}, authored by Rosiel Mendon\c{c}a, \textcolor{magenta}{Brazil}}}
	
	\cvitem{2016-06-22}{\textbf{Glamurama} ,\href{https://glamurama.uol.com.br/regina-miranda-prepara-danca-instalacao-para-ingles-ver-nas-olimpiadas/}{\textit{Regina Miranda prepara dança instalação para inglês ver na Olimpíada}}, \textcolor{magenta}{Brazil}}
	
	\section{Residencies}
	\cvitem{01/2022}{{\textbf{Artist in Resident:} at \href{https://zkm.de/en}{Center for Art and Media Karlsruhe}, as part of \href{https://onthefly.space/read/on-the-fly-live-coding-research-open-call-results}{on-the-fly} project sponsored by Creative Europe Programme of European Union, selected as an artist in resident out of 76 applications, in \textcolor{magenta}{Karlsruhe, Germany}}}
	\phantomsection
    \label{AR3}
	\cvitem{07/2020 10/2020}{\textbf{Artist in Resident (AR-3):} Virtual artist residency at \href{https://cinetic.arts.ro/evenimente/rezidentele-cinetic-2020/}{the International Center for Research and Education in Innovative and Creative Technologies (CINETic), National University of Theatre and Film “I.L. Caragiale” Bucharest (UNATC)}, \textcolor{magenta}{Bucharest, Romania}}	
	%\cvitem{03/2019}{\textbf{Visiting Researcher:} the  \href{https://www.hf.uio.no/ritmo/english/}{the Centre for Interdisciplinary Studies in Rhythm, Time and Motion (RITMO) at the University of Oslo} \textcolor{magenta}{Oslo, Norway}}
	\cvitem{12/2018}{\textbf{Artist in Resident:} Institute for Computer Music and Sound Technologies (ICST), Zurich University of Arts (ZHdK), \textcolor{magenta}{Zurich, Switzerland}}
	%\cvitem{05/2017}{\textbf{Visiting Researcher:}\href{https://www.cirmmt.org/}{ the Centre for Interdisciplinary Research in Music Media and Technology (CIRMMT)}, at \href{https://www.mcgill.ca/music/}{the Schulich School of Music at McGill University} \textcolor{magenta}{Montreal, QC, Canada}}
	%\cvitem{05/2016}{\textbf{Visiting Researcher: }\href{http://movingstories.ca/events/}{\textit{movement.futures} - May Residency 2016} at the Emily Carl University of Arts,  Vancouver, BC, Canada}
	\cvitem{07-09/2009}{\textbf{Artist in Resident:}\textit{ Bauhaus Project}, Bauhaus University in collaboration with the Middle East Tehnical University, \textcolor{magenta}{Weimar, Germany} and Ankara, Turkey}
	

	
	%\section{References}
	%\cvitem{1}{Philippe Pasquier - Associate Professor, School of Interactive Arts + Technology, Simon Fraser University - \href{mailto:pasquier@sfu.ca}{pasquier@sfu.ca}}
	%\cvitem{2}{Oliver Bown - Senior Lecturer, Art and Design, University of New South Wales - \href{mailto:o.bown@unsw.edu.au}{o.bown@unsw.edu.au}}
	%\cvitem{3}{Steve Dipaola - Graduate Chair: School of  Interactive Arts + Technology, Director: Cognitive Science Program, Simon Fraser University - \href{mailto:sdipaola@sfu.ca}{sdipaola@sfu.ca}}
	
	%----------------------------------------------------------------------------------------
	%	COVER LETTER
	%----------------------------------------------------------------------------------------
	
	% To remove the cover letter, comment out this entire block
	
	%\clearpage
	%
	%\recipient{HR Departmnet}{Corporation\\123 Pleasant Lane\\12345 City, State} % Letter recipient
	%\date{\today} % Letter date
	%\opening{Dear Sir or Madam,} % Opening greeting
	%\closing{Sincerely yours,} % Closing phrase
	%\enclosure[Attached]{curriculum vit\ae{}} % List of enclosed documents
	%
	%\makelettertitle % Print letter title
	%
	%\lipsum[1-3] % Dummy text
	%
	%\makeletterclosing % Print letter signature
	%
	%%----------------------------------------------------------------------------------------
	
\end{document}