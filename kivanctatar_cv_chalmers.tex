%%%%%%%%%%%%%%%%%%%%%%%%%%%%%%%%%%%%%%%%%
% "ModernCV" CV and Cover Letter
% LaTeX Template
% Version 1.1 (9/12/12)
%
% This template has been downloaded from:
% http://www.LaTeXTemplates.com
%
% Original author:
% Xavier Danaux (xdanaux@gmail.com)
%
% License:
% CC BY-NC-SA 3.0 (http://creativecommons.org/licenses/by-nc-sa/3.0/)
%
% Important note:
% This template requires the moderncv.cls and .sty files to be in the same 
% directory as this .tex file. These files provide the resume style and themes 
% used for structuring the document.
%
%%%%%%%%%%%%%%%%%%%%%%%%%%%%%%%%%%%%%%%%%

%----------------------------------------------------------------------------------------
%	PACKAGES AND OTHER DOCUMENT CONFIGURATIONS
%----------------------------------------------------------------------------------------

\documentclass[10pt,a4paper]{moderncv} % Font sizes: 10, 11, or 12; paper sizes: a4paper, letterpaper, a5paper, legalpaper, executivepaper or landscape; font families: sans or roman

\moderncvstyle{classic} % CV theme - options include: 'casual' (default), 'classic', 'oldstyle' and 'banking'
\moderncvcolor{grey} % CV color - options include: 'blue' (default), 'orange', 'green', 'red', 'purple', 'grey' and 'black'
\usepackage{lipsum}
\usepackage{amsmath}
\usepackage{url}
\usepackage[scaled]{beramono}
\renewcommand*\familydefault{\ttdefault} %% Only if the base font of the document is to be typewriter style
\usepackage[T1]{fontenc}

\usepackage[scale=0.87]{geometry} % Reduce document margins
\linespread{1.1}
\setlength{\hintscolumnwidth}{2.16cm} % Uncomment to change the width of the dates column
%\setlength{\makecvtitlenamewidth}{10cm} % For the 'classic' style, uncomment to adjust the width of the space allocated to your name

%----------------------------------------------------------------------------------------
%	NAME AND CONTACT INFORMATION SECTION
%----------------------------------------------------------------------------------------

\firstname{K{\i}van\c{c}} % Your first name
\familyname{Tatar} % Your last name

% All information in this block is optional, comment out any lines you don't need
\title{Assistant Professor in Interactive AI}
\address{Chalmers University of Technology \\Data Science and AI Division \\ Computer Science Department \\ Rännvägen 6B,\\}{  
412 58 Gothenburg, Sweden}

\email{tatar@chalmers.se}
\homepage{aicomparts.com}{Research Group} 
\webpage{kivanctatar.com}{Arts and Music} 

%----------------------------------------------------------------------------------------
\begin{document}
	%\ttfamily
	%\asciifamily
	\makecvtitle % Print the CV title

\cvitem{\today}{This CV consists of two sections. First, the \textit{CV Overview} provides statistics of achievements and accomplishments per CV category. Second, the \textit{Full CV} lists all CV items with their details for each CV category. The Full CV is organized according to the Chalmers CV guidelines decided in \textbf{C 2023-0548}, described at the link: \url{https://www.chalmers.se/en/about-chalmers/work-with-us/academic-positions/instructions-for-compiling-an-application/#guidelines-for-a-curriculum-vitae}.}

\cvitem{}{\begin{center}\LARGE CV Overview\end{center}} 
\hline
\section{Publications}
\cvitem{Google Scholar}{\url{https://scholar.google.com/citations?user=jLujPGQAAAAJ}}
%\cvitem{Metrics}{\href{https://scholar.google.com/citations?user=jLujPGQAAAAJ&hl=sv&oi=ao}{h-index 9 with 307 total citations on google scholar.} if numbers matter.}

%  Please note that music technology is a discipline with relatively low number of total publications; thus the h-index should be assessed according to the size of the discipline.

\cvitem{Journal Publications}{Since October 2021, I have published three journal papers: one journal paper as the first author on Leonardo (MIT Press), another on Organized Sound (Cambridge Press) as the second author, and another as the last author is accepted with major revisions at Humanities and Social Sciences Communications (Springer Nature).}

\cvitem{Conference Publications}{Since October 2021, I have published 9 conference papers, one as the first author, and the others as a co-author, at conferences Sound and Music Computing Conference, New Interfaces for Musical Expression Conference, Nordic-CHI conference, ACM conference on Movement Computing, and AI Music Creativity conference.}

%\cvitem{Books}{I have received an invitation to write a book on Musical AI from the publisher Routledge in UK. This is planned to be completed late 2025.}

\cvitem{Collabs}{The publications above included local, national, and international collaborations.}

\section{Teaching}

\cvitem{Curriculm Development}{I have initiated and carried out the first run of a new course TRA 385 - Emerging Technologies through Artistic Innovation at Chalmers. Previously, I have been teaching various courses covering topics in Sound Design, Machine Learning and AI, Interaction Design, Electronics and Prototyping, and Art and Technology.}

\cvitem{Pedagogical Training}{My pedagogical training is finalized as of June 2024 including PIL 101, PIL 102, PIL 103, PIL 201 from Gothenburg University, and CLS 930 from Chalmers.}

\cvitem{Supervision}{I am currently the main supervisor of Kelsey Cotton (PhD student) and Xuechen Liu (Post-doctoral Fellow). I was one of the assistant supervisors of Georgios Diapoulis who graduated in Fall 2024. Lastly, I supervised 5 master thesis, currently supervising two thesis in 2024, and I supervised three masters summer interns.2}

\section{Outreach}

\cvitem{Research Outreach}{Since 2021, I have shared my research in public events of workshops, seminars, and panels at 12 events in 7 countries in 2 continents.}

\cvitem{Artistic Outreach}{Since 2021, I have performed or exhibited my artworks at 10 public events in 7 countries in 3 continents, reaching over 60k in-person visitors/audience, with press coverage.}

\section{Academic Citizenship}

\cvitem{Organization}{Since October 2021, I have organized two PhD seminars in Gothenburg and Stockholm funded by WASP-HS cross-collaboration grant. I have co-organized a workshop at Nordic-CHI. I was the Arts Chair for TEI 2023 and we organized an arts exhibition at this conference. I am the industry chair of AIMC 2024.}

\cvitem{Reviewing}{Since October 2021, I have reviewed 24 full conference papers and 6 short papers, and 3 journal paper; at the venues International Conferences on Computational Creativity, AI Music Creativity conferences, International Conferences on New Interfaces for Musical Expression, EvoMUSART, the journal Personal and Ubiquitous Computing (Springer Press), Leonardo Journal (MIT Press), and Neural Computing and Applications (Springer).}

\cvitem{Thesis Examination}{I was the external examiner of a master thesis on music technology at University of Oslo in 2022.}

\cvitem{}{\begin{center}\LARGE Full CV\end{center}}
\hline

    \section{Contact Information}

    \cvitem{Address}{Chalmers University of Technology, Data Science and AI Division, Computer Science Department, Rännvägen 6B, 412 58 Gothenburg, Sweden}   
    \cvitem{Email}{\href{mailto:tatar@chalmers.se}{tatar@chalmers.se}}
    \cvitem{Phone}{+46 72 1779088}
    \cvitem{Research Group}{\href{https://aicomparts.com}{aicomparts.com}} 
    \cvitem{Personal Page}{\href{https://kivanctatar.com}{kivanctatar.com}} 
	\cvitem{ORCID}{\href{https://orcid.org/0000-0003-4133-8641}{0000-0003-4133-8641}} 
    \cvitem{Google Scholar}{\url{https://scholar.google.com/citations?user=jLujPGQAAAAJ}} 

    \section{Work Experience}
	\cvitem{ongoing 2021-10}{\textbf{Assistant Professor} in Interactive AI; Data Science and AI division, Computer Science and Engineering, Chalmers University of Technology; Gothenburg, Sweden}
	\cvitem{04-2020 10-2019}{\textbf{Post-Doctoral Researcher}, Institute for Computer Music and Sound Technologies, Zurich University of Arts, Switzerland}
	\cvitem{2020-08 2019-09}{\textbf{Post-Doctoral Fellow}, (part-time) Metacreation Lab, Simon Fraser University, Greater Vancouver, BC Canada}
	\cvitem{2019-03 2014-09}{\textbf{Research Assistant}, Metacreation Lab, Simon Fraser University, Greater Vancouver, BC, Canada}
	\cvitem{2011 Summer}{\textbf{Intern Engineer} Turkish Radio and Television Foundation, Ankara}
	\cvitem{2011-01 2009-09}{\textbf{Student Assistant}, Fine Arts and Music Department, METU, Ankara, Turkey}
	\cvitem{2010 Summer}{\textbf{Intern Engineer}, Esenbo\u{g}a Airport, Ankara, Turkey }
	%\cvitem{Summer 2009}{\textbf{Project Assistant}, \textit{Bauhaus Impact} Fine Arts and Music Department, METU (Ankara, Turkey) and Bauhaus University (Weimar, Germany)}

 
 \section{Educational Qualifications }
	\cvitem{2019-07 2014-09}{Ph.D., Interactive Arts and Technology (SIAT), Simon Fraser University (SFU), Greater Vancouver, BC Canada. Supervisors: Philippe Pasquier (Main), Steve Dipaola, and Oliver Bown. Thesis title: Musical agents based on self-organizing maps for audio applications} 
	\cvitem{2012--2014}{M.Mus., Sonic Arts, Center for Advanced Studies in Music (MIAM), \.{I}stanbul Technical University (ITU), \.{I}stanbul, Turkey. Supervisors: Reuben de Latour and Anil Camci.}
	%; GPA -- 3.58/4.00} 
	\cvitem{2012--2006}{B.Sc., Electrical and Electronics Engineering, Middle East Technical University (METU), \textit{Ankara, Turkey}}
	%; GPA -- 3.03/4.00}

    \section{Deductible Time}
    \cvitem{}{I have not taken any parental or extended sick leave.}

    \section{Research Grants}
    \cvitem{}{\small PI: Principal Investigator, CI: Co-Investigator, CA: Co-Applicant, CL: Collaborator}
    \cvitem{}{\small NSERC: Natural Sciences and Engineering Research Council of Canada}
    \cvitem{}{\small SSHRC: Social Sciences and Humanities Research Council of Canada}
    \cvitem{}{\small CCA: Canada Council for the Arts}
    \cvitem{}{\small CHAIR: Chalmers AI Research Centre}
    
    {\begin{center}\small\textcolor{magenta}{\textcolor{magenta}{Submitted Projects (waiting Decision)}}\end{center}}

    \cvitem{SSF-1}{\href{https://strategiska.se/en/}{Swedish Foundation for Strategic Research}, \href{https://strategiska.se/en/apply-for-ssf-future-research-leader-2/}{Future Research Leaders call}, K{\i}van\c{c} Tatar (PI). 16 million SEK }

   \cvitem{VR-1}{\href{https://vr.se/en/}{Vetenskapsrådet}, \href{https://www.vr.se/english/applying-for-funding/calls/2023-11-15-research-environment-grant---humanities-and-social-sciences.html}{Research environment grant – humanities and social sciences}, K{\i}van\c{c} Tatar (PI). 18 million SEK }
    
    \cvitem{}{\begin{center}\small\textcolor{magenta}{\textcolor{magenta}{Ongoing Projects}}\end{center}}
    
    \cvitem{WASPHS-1}{\href{https://wasp-hs.org/people/kivanc-tatar}{\textit{Interactive AI – Ethics and Aesthetics of Human-Machine Interaction in Art, Music, and Games}}; kindly offered to K{\i}van\c{c} Tatar (PI) as a starting package for Assistant Professorship position. 12 million SEK}
    \cvitem{WASPHS-2}{\href{https://wasp-hs.org/people/kivanc-tatar}{Cross-collaboration grant}; K{\i}van\c{c} Tatar (PI), Morten Fjeld (Co-PI), Andre Holzapfel (Co-PI). 100 000 SEK}
        
    \cvitem{}{\begin{center}\small\textcolor{magenta}{\textcolor{magenta}{2022}}\end{center}}
    
    \cvitem{CHAIR-1}{CHAIR X + AI Call- \href{https://www.chalmers.se/en/centres/chair/opportunities/Pages/Call-for-projects-CHAIR-X---AI.aspx}{\textit{AI + Social Drones: Towards Autonomous and Adaptive Social Drones}.} Mohammad Obaid (PI), K{\i}van\c{c} Tatar (co-PI), Mikael Wiberg (co-PI).  400 000 SEK}
    
    \cvitem{}{\begin{center}\small\textcolor{magenta}{\textcolor{magenta}{2021}}\end{center}}
    
    
    \cvitem{CCA-10}{CCA - Explore and Create - Concept to Realization. K{\i}van\c{c} Tatar (PI); with support from Paul Paroczai, Esra \"Ozkan, Nancy Lee, and Lucy Strauss. CAD \$51505 }
    \phantomsection
    \label{CCA9}
    \cvitem{CCA-9}{CCA - Explore and Create - Concept to Realization. K{\i}van\c{c} Tatar (PI); with support from Tamar Tabori, Dan O'Shea, and Remy Siu. CAD \$23100}
    \phantomsection
    \label{BCAC1}
    \cvitem{BCAC-1}{BC Arts Council - Pivot for Individuals - Professional Development. K{\i}van\c{c} Tatar (PI); with support from Paul Paroczai and Remy Siu. CAD \$12000 }
    \phantomsection
    \label{CCA8}
    \cvitem{CCA-8}{CCA - Explore and Create - Research and Creation. K{\i}van\c{c} Tatar (PI). CAD \$19500 }
    \cvitem{}{\begin{center}\small\textcolor{magenta}{\textcolor{magenta}{2020}}\end{center}}
    \phantomsection
    \label{CCA7}
    \cvitem{CCA-7}{CCA - Research and Creation - Remy Siu (PI, 1st author), K{\i}van\c{c} Tatar (Creative AI consultant, developer).  CAD \$37700 CAD}
    \cvitem{CCA-6}{CCA -  Strategic Funds and Initiatives, COVID-19 Emergency Support Fund. CAD \$ 5000} 
    \phantomsection
    \label{CCA5}
    \cvitem{CCA-5}{CCA - Concept to Realization - Liz Solo (PI, 1st author), Dr. Jeremy O Turner (PI, 2nd author), Mike Kean (composer), K{\i}van\c{c} Tatar (Creative AI consultant, developer).  CAD \$58930}
    \cvitem{CC-1}{Canada Compute - Research Allocation -  Philippe Pasquier (PI), K{\i}van\c{c} Tatar (1st author), Jeff Ens (2nd author), Omid Alemi (3rd author) computing resources of 60 CPU years and 3 GPU years, with estimated worth of CAD \$14,588}
    \cvitem{}{\begin{center}\small\textcolor{magenta}{\textcolor{magenta}{2019}}\end{center}}
    
    \cvitem{SNSF-1}{Swiss National Science Foundation (SNF): Research Exchange - K{\i}van\c{c} Tatar (CA, Visiting Researcher, 1st author), Daniel Bisig (PI), Philippe Kocher (CI), Martin Neukom (CI), Germ\`an Toro-P\`erez (CI) - CHF 19500}
    \cvitem{CCA-4}{SSHRC Connection: Marcelo Wanderlay (PI), Pascal Baltazar (CL, 1st author), Philippe Pasquier (CI), K{\i}van\c{c} Tatar (CL, 2nd author) - CAD \$ 24900  }
    \phantomsection
    \label{CCA3}
    \cvitem{CCA-3}{CCA - Arts Abroad: Travel - K{\i}van\c{c} Tatar (PI), Philippe Pasquier (CA) - CAD \$3950}
    \phantomsection
    \label{CCA2}
    \cvitem{CCA-2}{CCA - Arts Abroad : Residencies - K{\i}van\c{c} Tatar (PI), Philippe Pasquier (CA) - CAD \$7800}
    \cvitem{}{\begin{center}\small\textcolor{magenta}{\textcolor{magenta}{2018}}\end{center}}
    \cvitem{CCA-1}{CCA - Arts Across Canada: Travel - K{\i}van\c{c} Tatar (PI), Philippe Pasquier (CA), Remy Siu (CA) - CAD \$5100}
    \cvitem{NSERC-1}{NSERC Engage, Canada - Philippe Pasquier (PI), K{\i}van\c{c} Tatar (1st author), \href{http://www.tangibleinteraction.com/}{Tangible Interaction (industry partner)} CAD \$15000}
    \cvitem{}{\begin{center}\small\textcolor{magenta}{\textcolor{magenta}{2017}}\end{center}}
    \cvitem{SSHRC-1}{SSHRC Small Grant - Philippe Pasquier (PI), Mirjana Prpa (CA), K{\i}van\c{c} Tatar (CA), Bernhard Riecke (CI) - CAD \$7000}

\section{Supervision}

\cvitem{}{\begin{center}\small\textcolor{magenta}{\textcolor{magenta}{Post-doctoral Fellows}}\end{center}}

\cvitem{Ongoing}{\href{https://scholar.google.com/citations?user=ni5MM4oAAAAJ}{\textbf{Xuechen (Hugh) Liu}}. \textit{Postdoc in Interactive AI for interdisciplinary Artistic Practices}. 2023-01::2026-01. \textcolor{magenta}{Supervisor: K{\i}van\c{c} Tatar}. }
\cvitem{}{\begin{center}\small\textcolor{magenta}{\textcolor{magenta}{Ph.D. students}}\end{center}}

\cvitem{Ongoing}{\href{https://kelseycotton.com/}{\textbf{Kelsey Cotton}}. \textit{Interactive Music Systems using Artificial Intelligence}. Expected graduation 2027. \textcolor{magenta}{Main Supervisor: K{\i}van\c{c} Tatar}, Co-supervisor: Devdatt Dubhashi, Examiner: Graham Kemp. Funded by WASP-HS-1.}

\cvitem{Alumni}{\href{https://scholar.google.com/citations?user=Lh4Oz9EAAAAJ}{\textbf{Georgios Diapoulis}}. \textit{Bottom-up Live Coding}. Graduated in Fall 2023. Main Supervisor: Palle Dahlsted, Co-supervisor: Mohammad Obaid, \textcolor{magenta}{Co-supervisor: K{\i}van\c{c} Tatar}, Examiner: Staffan Bj\"ork.}

\cvitem{}{\begin{center}\small\textcolor{magenta}{\textcolor{magenta}{Master students}}\end{center}}

\cvitem{2023}{Priscilla Tissot.\textit{A Framework for Exposing Bias in Generative Deep Learning models for Image and Video Applications}. \textcolor{magenta}{Main Supervisor: K{\i}van\c{c} Tatar}. Summer Intern.}

\cvitem{2024}{Antonio Mangoni Di S Stefano. \textit{Neuro-Symbolic Creation of Non-Playable Characters}, Examiner: Morten Fjeld.}

\cvitem{2024}{Filip Lundström. \textit{Incorporating Brain-Computer Interface (Muse) with Deep Learning Audio Synthesis}, Examiner: Palle Dahlsted.}

\cvitem{2024}{Matteo Caravati.\textit{Interfacing ErgoJr with Creative Coding Platforms}. \textcolor{magenta}{Main Supervisor: K{\i}van\c{c} Tatar}. Summer Intern.}

\cvitem{2023}{Nicolas Gry.\textit{Embedding Deep Learning Synthesis in Ossia framework}. \textcolor{magenta}{Main Supervisor: K{\i}van\c{c} Tatar}. Summer Intern.}

\cvitem{2023}{Don Sameera Parakrama Ratnayake. \textit{Immersive Tactile Steering Interface: A bio-feedback interface for steering devices}. \textcolor{magenta}{Main Supervisor: K{\i}van\c{c} Tatar}, Examiner: Palle Dahlsted.}

\cvitem{2023}{Anton Eriksson. \textit{Applying Affect Estimation to
3D Music Visualization}. \textcolor{magenta}{Main Supervisor: K{\i}van\c{c} Tatar}, Examiner: Palle Dahlsted.}

\cvitem{2023}{David H\"ogberg. \textit{Latent Vector Synthesis}. \textcolor{magenta}{Main Supervisor: K{\i}van\c{c} Tatar}, Examiner: Staffan Bj\"ork.}
\cvitem{2023}{Chaoming Wang. M\textit{Machine Learning for generative
painting informed by visual arts}. \textcolor{magenta}{Main Supervisor: K{\i}van\c{c} Tatar}, Examiner: Palle Dahlsted.}
\cvitem{}{Lin Luo and Zixi Geng. \textit{The EaseFit:
An Interactive Sonic Design with E-textile}. \textcolor{magenta}{Main Supervisor: K{\i}van\c{c} Tatar}, Co-Supervisor: Palle Dahlsted, Examiner: Staffan Bj\"ork.}

\section{Teaching Experience}

\cvitem{2025 Spring}{\textbf{Examiner.} Advanced course. TRA 385 - Machine Learning and AI through artistic innovation, upcoming, Chalmers University of Technology, Gothenburg, Sweden.}

\cvitem{2024 Spring}{\textbf{Examiner.} Advanced course. \href{https://chalmers.instructure.com/courses/29250}{TRA 385 - Emerging technologies through artistic innovation}, 18 students, Chalmers University of Technology, Gothenburg, Sweden.}

\cvitem{2023 Fall}{\textbf{Co-Teacher}. Advanced course.\href{https://chalmers.instructure.com/courses/25338}{CIU265 Interaction design project / DAT375/DIT460 Game Development Project}, 75 students, Chalmers University of Technology, Gothenburg, Sweden.}

\cvitem{2023 Fall}{\textbf{Co-Teacher}. Advanced course. \href{https://chalmers.instructure.com/courses/25338}{CIU176 / TIA108 Prototyping in interaction design}, 68 students, Chalmers University of Technology, Gothenburg, Sweden}

\cvitem{2022 Fall}{\textbf{Co-Teacher}. Advanced course. \href{https://student.portal.chalmers.se/sv/chalmersstudier/minkursinformation/Sidor/SokKurs.aspx?course_id=25655&parsergrp=3}{CIU281 - Emerging trends and critical topics in interaction design}, 37 students, Chalmers University of Technology, Gothenburg, Sweden}

\cvitem{2022 Spring}{\textbf{Co-Teacher}. Advanced course. \href{https://student.portal.chalmers.se/en/chalmersstudies/courseinformation/pages/searchcourse.aspx?course_id=22017&parsergrp=3}{CIU180 / TIA107 Tangible interaction}, 40 students, Chalmers University of Technology, Gothenburg, Sweden}
\cvitem{2021 Fall}{\textbf{Co-Teacher}, \href{https://student.portal.chalmers.se/sv/chalmersstudier/minkursinformation/Sidor/SokKurs.aspx?course_id=25655&parsergrp=3}{CIU281 - Emerging trends and critical topics in interaction design}, 37 students, Chalmers University of Technology, Gothenburg, Sweden}

\cvitem{2018 Spring}{\textbf{Teaching Assistant}. Bachelor level course. \textit{IAT 340 - Sound Design}, 47 students, SIAT, Simon Fraser University, Vancouver, BC Canada}

\cvitem{2017-2019}{\textbf{Teaching Assistant}. Massive Open Online Course (MOOC). \textit{\href{https://www.kadenze.com/courses/advanced-generative-art-and-computational-creativity/}{Kadenze: Advance Generative Art and Computational Creativity}}}
\cvitem{2016-2019}{\textbf{Teaching Assistant}. Massive Open Online Course. \textit{\href{https://www.kadenze.com/courses/generative-art-and-computational-creativity/}{Kadenze: Generative Art and Computational Creativity}}}

\cvitem{2017 Spring}{\textbf{Teaching Assistant}. Bachelor level course. \textit{IAT 340 - Sound Design}, 50 students, SIAT, Simon Fraser University, Vancouver, BC Canada}
\cvitem{2016 Spring}{\textbf{Sessional Instructor} (Examiner). Bachelor level course. \textit{FPA 149 - Sound}, 135 students, School of Contemporary Arts, Simon Fraser University, Vancouver, BC Canada}
\cvitem{2015 Fall}{\textbf{Teaching Assistant}. Bachelor level course. \textit{IAT 222 - Interactive Arts}, SIAT, Simon Fraser University, Vancouver, BC Canada}
\cvitem{2013-2014}{\textbf{Lecturer}. Professional skill course. SAE Institue \.{I}stanbul, Courses: \textit{Basic Electronics, Studio Cabling and Patchbays, dB\&Metering}, \.{I}stanbul, Turkey}

\section{Pedagogy Training}

\cvitem{2024}{PIL 103: Teaching and Learning in Higher Education 3. Gothenburg University.}
\cvitem{2023}{CLS 930: Diversity and Inclusion for Learning in Higher Education. Chalmers University of Technology.}
\cvitem{2023}{PIL 201: Supervision in Postgraduate Programmes. Gothenburg University.}
\cvitem{2023}{PIL 102: Teaching and Learning in Higher Education 2. Gothenburg University.}
\cvitem{2023}{PIL 101: Teaching and Learning in Higher Education 1. Gothenburg University.}

\section{Academic Citizenship - Reviewing for Funding Agencies and Other Universities}
\cvitem{2024}{\textbf{Jury duty}. Reviewed a grant application for the KU Leuven Research Council in Belgium.}
\cvitem{2023}{\textbf{Jury duty}. Reviewed 96 grant applications for Concept to Realization Track at the Canada Council for the Arts.}

\section{Academic Citizenship - Conference Organization}

\cvitem{2024}{\textbf{Co-Organizer}. AIMC 2024, AI Music Creativity Conference 2024, Oxford, UK.}

\cvitem{2023}{\textbf{Co-Organizer}. TEI 2023, The ACM International Conference on Tangible, Embedded and Embodied Interaction, Warsaw, Poland.}

\cvitem{2022}{\textbf{Co-PI, Co-Organizer}. Social Drones for Health and Well-being Workshop at Nordic CHI 2022, organized with Mohammad Obaid and Michael Wiberg, funded by CHAIR-1, held in Aarhus, Denmark.} 

\cvitem{2022}{\textbf{PI, Co-Organizer}. \href{https://aicomparts.com/projects/2022-phd-seminar/}{PhD Seminar in Interdisciplinary Research in Human-Computer Interaction}, funded by WASPHS-2, organized with Andre Holzapfel, at KTH, Stockholm, Sweden.} 

\cvitem{2022}{\textbf{PI, Co-Organizer}. \href{https://aicomparts.com/projects/2022-phd-seminar/}{PhD Seminar in Interdisciplinary Research in Human-Computer Interaction, funded by WASPHS-2, organized with Morten Fjeld and Andre Holzapfel, at Chalmers University of Technology, Gothenburg, Sweden.} 

\cvitem{2019}{\textbf{Co-Organizer}. Musical Metacreation Workshop 2019, \href{http://musicalmetacreation.org/mume-2019}{MUME 2019}, in conjunction with the Tenth International Conference on Computational Creativity, \href{http://www.computationalcreativity.net/iccc2019/}{ICCC 2019}, North Carolina, Charlotte, USA.}

\cvitem{2018}{\textbf{Co-Organizer}. Musical Metacreation Workshop 2018, \href{http://musicalmetacreation.org/mume-2018}{MUME 2018}, in conjunction with the Ninth International Conference on Computational Creativity, \href{http://www.computationalcreativity.net/iccc2018/}{ICCC 2018}, Salamanca, Spain.}

\cvitem{2017}{\textbf{Co-Organizer}. Musical Metacreation Workshop 2017, \href{http://musicalmetacreation.org/mume-2017}{MUME 2017}, in conjunction with the Eighth International Conference on Computational Creativity, \href{http://www.computationalcreativity.net/iccc2017/}{ICCC 2017}, Atlanta, Georgia, US.}

\cvitem{2016}{\textbf{Co-Organizer}. Musical Metacreation Workshop 2016, \href{http://musicalmetacreation.org/mume-2016}{MUME 2016}, in conjunction with the Seventh International Conference on Computational Creativity, \href{http://www.computationalcreativity.net/iccc2016/}{ICCC 2016}, Paris, France.}

\section{Academic Citizenship - Reviewing Activities for Journals and Conferences}

\cvitem{2024}{\textbf{Reviewer}. Neural Computings and Applications, Springer Press, 1 paper.}
\cvitem{2024}{\textbf{Reviewer}. Leonardo, MIT Press, 1 paper.}

\cvitem{2024}{\textbf{Program Committee Member}. \href{http://computationalcreativity.net/iccc24/}{ICCC 2024, The International Conference on Computational Creativity 2024}, 3 long and 3 short papers.}
\cvitem{2024}{\textbf{Program Committee Member}. {IJCAI 2024, the 33rd International Joint Conference on Artificial Intelligence}, 2 long papers.}
\cvitem{2023}{\textbf{Program Committee Member}. \href{http://computationalcreativity.net/iccc23/}{ICCC 2023, The International Conference on Computational Creativity 2022}, 3 long papers, 3 short paper.}

\cvitem{2023}{\textbf{Program Committee Member}. \href{http://computationalcreativity.net/iccc22/}{ICCC 2023, The International Conference on Computational Creativity 2022}, 3 long papers, 3 short paper.}
\cvitem{2023}{\textbf{Arts Chair}. TEI 2023, The ACM International Conference on Tangible, Embedded and Embodied Interaction, reviewed all arts submissions and curated an exhibition.}
\cvitem{2022}{\textbf{Program Committee Member}. \href{https://2022.aimusiccreativity.org/}{AIMC 2022, AI Music Creativity 2022}, 3 papers.}
\cvitem{2022}{\textbf{Program Committee Member}. \href{http://computationalcreativity.net/iccc22/}{ICCC 2022, The International Conference on Computational Creativity 2022}, 3 papers.}
\cvitem{2022}{\textbf{Program Committee Member}. \href{https://nime2022.org/index.html}{NIME 2022, The International Conference on New Interfaces for Musical Expression 2022}, 3 papers.}
\cvitem{2022}{\textbf{Reviewer.} \href{https://www.springer.com/journal/779}{Personal and Ubiquitous Computing}, Springer Press, 1 paper.}	
\cvitem{2022}{\textbf{Program Committee Member}. \href{http://www.evostar.org/2021/evomusart/}{EvoMUSART 2022, the “10th International Conference on Artificial Intelligence in Music, Sound, Art and Design”}, 3 papers.}
\cvitem{2021}{\textbf{Reviewer.} \href{https://www.mdpi.com/journal/mathematics}{MDPI Mathematics - Special Issue ``Mathematics and Computation in Music''}, 1 paper.}
\cvitem{2021}{\textbf{Program Committee Member}. \href{http://www.evostar.org/2021/evomusart/}{EvoMUSART 2021, the “10th International Conference on Artificial Intelligence in Music, Sound, Art and Design”}, 3 papers.}
\cvitem{2021}{\textbf{Program Committee Member}. \href{http://computationalcreativity.net/iccc21/}{the 12th International Conference on Computational Creativity}. 3 long and 3 short papers.}
\cvitem{2020}{\textbf{Reviewer}. \href{https://www.springer.com/journal/521}{Neural Computing and Applications}, 1 paper.}
\cvitem{2020}{\textbf{Reviewer}. \href{https://www.tandfonline.com/toc/hihc20/current}{International Journal of Human-Computer Interaction}, 1 paper.}	
\cvitem{2020}{\textbf{Program Committee Member}. the Tenth International Conference on Computational Creativity, \href{http://www.computationalcreativity.net/iccc2020/}{ICCC 2020}, reviewed 3 long papers and 3 short papers.}

\cvitem{2020}{\textbf{Reviewer}. \href{https://networkmusicfestival.org/organising-commitee/}{Network Music Festival 2020}, 3 artworks.}

\cvitem{2019}{\textbf{Reviewer}. \href{https://www.mitpressjournals.org/lmj}{Leonardo Music Journal, MIT Press}, 1 paper.}

\cvitem{2020-2019}{\textbf{Reviewer}. \href{https://www.tandfonline.com/loi/nnmr20}{Journal of New Music Research, Taylor and Francis}, 2 papers.}

\cvitem{2019}{\textbf{Program Committee Member}. the Ninth International Conference on Computational Creativity, \href{http://www.computationalcreativity.net/iccc2019/}{ICCC 2019} , 3 papers.}

\cvitem{2019}{\textbf{Co-Organizer, Program Committee Member}. Musical Metacreation Workshop 2019, \href{http://musicalmetacreation.org/mume-2019}{MUME 2019}, in conjunction with the Tenth International Conference on Computational Creativity, \href{http://www.computationalcreativity.net/iccc2019/}{ICCC 2019}.}

\cvitem{2018}{\textbf{Co-Organizer, Chair}. Musical Metacreation Workshop 2018, \href{http://musicalmetacreation.org/mume-2018}{MUME 2018}, in conjunction with the Ninth International Conference on Computational Creativity, \href{http://www.computationalcreativity.net/iccc2018/}{ICCC 2018}, reviewed 1 paper.}

\cvitem{2017}{\textbf{Reviewer}. The Genetic and Evolutionary Computation Conference 2017 \href{http://gecco-2017.sigevo.org/}{GECCO 2017}, reviewed 1 paper.}

\cvitem{2017}{\textbf{Subreviewer} International Joint Conference on Artificial Intelligence 2017 \href{https://ijcai-17.org/}{IJCAI 2017}.}

\cvitem{2017}{\textbf{Co-Organizer, Chair}. Musical Metacreation Workshop 2017, \href{http://musicalmetacreation.org/mume-2017}{MUME 2017}, in conjunction with the Eighth International Conference on Computational Creativity, \href{http://www.computationalcreativity.net/iccc2017/}{ICCC 2017}, reviewed 5 papers.}

\cvitem{2017}{\textbf{Subreviewer}. \href{https://dl.acm.org/citation.cfm?id=3023311&picked=prox&CFID=962683035&CFTOKEN=33139292}{Computers in Entertainment (CIE) - Special Issue on Musical Metacreation Volume 14 Issue 2-3, Summer 2016}, reviewed 1 paper.}

\cvitem{2016}{\textbf{Co-Organizer, Chair}. Musical Metacreation Workshop 2016, \href{http://musicalmetacreation.org/mume-2016}{MUME 2016}, in conjunction with the Seventh International Conference on Computational Creativity, \href{http://www.computationalcreativity.net/iccc2016/}{ICCC 2016}, reviewed 9 papers}

\cvitem{2015}{\textbf{Subreviewer}. The Sixth International Conference on Computational Creativity, \href{http://www.computationalcreativity.net/iccc2015/}{ICCC 2015}.}

\section{Invited Presentations, Workshops, and Lectures}

Tatar, K. is the leading author of the following public presentations unless otherwise is indicated. \\

\cvitem{2024-03-20}{\textbf{Lecture.} \href{https://www.youtube.com/watch?v=HyRaBvoLvnI}{\textit{Organized Sound Spaces with Machine Learning}}, at \href{https://www.hfmt-hamburg.de/}{Hochschule für Musik und Theater Hamburg} in Germany, as a part of \href{https://www.youtube.com/playlist?list=PLlVs3aa4vQLpU4ni0sQExbGBT5YhKyWpg}{\_MUTOR Online Lecture Series - Artificial Intelligence for Music and Multimedia}.}

\cvitem{2024-03-20}{\textbf{Keynote.} \href{https://cinetic.arts.ro/en/evenimente/ai-in-art-practices-and-research-conference/}{AI in Art Practices and Research Conference.} \href{https://cinetic.arts.ro/en/about/}{CINETic - Centrul Internațional de Cercetare și Educație în Tehnologii Inovativ Creative},  I.L. Caragiale” National University of Theatre and Film in Bucharest, Romania.}

\cvitem{2023-11-15}{\textbf{Panel.}At WASP-HS - AI for Humanity and Society 2023, with Irina Shklovski, Guest Professor of Communication and Computing at Linköping University, Barry Brown Professor of Human-Computer Interaction at Stockholm University, Anne Kaun, Professor in Media and Communication Studies at Södertörn University. Malmo, Sweden}

\cvitem{2023-05-03}{\textbf{Seminar.} \href{https://www.communication.aau.dk/un-disciplinary-e73485#3rd--may}{\textit{Notes on Machine Learning and Artificial Intelligence for Artistic Practices}} as a part of \href{https://www.communication.aau.dk/un-disciplinary-e73485#3rd--may}{Un-Disciplinary} at Aalborg University in \textcolor{magenta}{Denmark}.}

\cvitem{2022-09-14 2022-09-17}{\textbf{Workshop.} \href{http://www.emutelab.org/blog/robot_opera}{\textit{Robots, AI, and Opera workshop}} at Aalborg University in \textcolor{magenta}{Denmark}.}

\cvitem{2022-09-13}{\textbf{Seminar.} \href{https://www.aau.dk/ai-and-creativity-e36349}{\textit{A Different Take on AI: Design Processes and Interaction in Machine Learning and Artificial Intelligence for Music and Arts}}, as a part of \textit{AI and Creativity: Exploring the Future of Music, Opera, and the Performing Arts in the Age of Machine Learning} at Aalborg University in \textcolor{magenta}{Denmark}.}

\cvitem{2022-06}{\textbf{Lecture}. \textit{An Introduction to Arts and Technology}. HCI Summer School 2022, organized by SIGCHI, Chalmers University of Technology, and Politechnika  Łódźka, hosted in \textcolor{magenta}{Łódź, Poland}}

\cvitem{2022-06}{\textbf{Workshop}.\textit{Sound Design using Deep Learning}. ACM HCI Summer School 2022, organized by SIGCHI, Chalmers University of Technology, and Politechnika  Łódźka, hosted in \textcolor{magenta}{Łódź, Poland}}

\cvitem{2022-03-31}{\textbf{Seminar and Panel}. \href{https://www.creative-ai-project.se/seminars/}{\textit{Musical Artificial Intelligence Architectures with Unsupervised Learning in Improvisation, Audio-Visual Performance, Interactive Arts, Dance, and Live Coding}} at the event \href{https://www.creative-ai-project.se/seminars/}{\textit{Interaction with generative music frameworks}}. As a part of seminar series \textit{dialogues: probing the future of creative technology} hosted virtually by \href{https://www.creative-ai-project.se}{Creative-Ai research group} of KTH Royal Institute of Technology, \textcolor{magenta}{Stockholm, Sweden}.}

\cvitem{2022-03-26}{\textbf{Panel}. Exploring links between body/image. K{\i}van\c{c} Tatar and Maria Pisiou. As a part of \href{https://4bidgallery.wordpress.com/2017/01/29/b-base/}{b.base expo} with Maria Pisiou at \href{https://4bidgallery.wordpress.com}{4bid Gallery} at OT301 in \textcolor{magenta}{Amsterdam, Netherlands}, invited through curatorial selection by 4bid gallery.}

\cvitem{2021-12}{\textbf{Seminar}. Incorporating Artificial Intelligence Architectures into Artistic Production and Live Performances, as a part of \href{https://www.sacmmt.com/}{Bowed Electrons 2021} hosted from \textcolor{magenta}{Cape Town, South Africa.}}

\cvitem{2021-10-15}{\textbf{Seminar}. Incorporating Artificial Intelligence Architectures into Artistic Production and Live Performances, at \href{https://kivanctatar.com/AI-Music-and-Improvisation-2021}{Pre-Conference - AI, Music and Improvisation} as a part of the conference Improvisation, Ecology and Digital Technology, hybrid format, in \textcolor{magenta}{Dusseldorf, Germany}}
\cvitem{2021-07-04}{\textbf{Artist Talk} on AI and Music Technology, with \href{https://remysiu.com/}{Remy Siu}, Yang Yi-Hsuan, and Su Li as a part of collective exhibition \href{https://event.culture.tw/NTMOFA/portal/Registration/C0103MAction?useLanguage=en&actId=10050&request_locale=en}{\textit{Aethereal}}, at the \href{https://www.ntmofa.gov.tw/en/}{Taiwan National Museum of Fine Arts}  \textcolor{magenta}{Taichung, Taiwan}}

\cvitem{2021-06-26}{\textbf{Workshop} on Creative Artificial Intelligence and Reinforcement Learning for Co-Creation, with \href{https://remysiu.com/}{Remy Siu}, as a part of collective exhibition \href{https://event.culture.tw/NTMOFA/portal/Registration/C0103MAction?useLanguage=en&actId=10050&request_locale=en}{\textit{Aethereal}}, at the \href{https://www.ntmofa.gov.tw/en/}{Taiwan National Museum of Fine Arts}  \textcolor{magenta}{Taichung, Taiwan}}
\cvitem{2020}{\textbf{Panel:} UNATC Distant Art CINETic Residencies. Tatar, K.; Tragtenberg, T.; Singer, M. O.; Hein, L. N.; and Schmitz C. \textcolor{magenta}{Ars Electronica 2020}, in \textcolor{magenta}{Linz, Austria}.}	

\cvitem{2020}{\textbf{Review Paper:} Review of Concert 12. \textit{Array: Special Issue - ICMC/NYCEMF 2019/ Reports+Reviews}. International Computer Music Association.}

\cvitem{2019-11}{\textbf{Workshop} on Creative Artificial Intelligence for Music and Multimedia, \href{https://cinetic.arts.ro/en/about/}{International Center for Research and Education in Innovative and Creative Technologies (CINETic), University of Theatre and Film ”I.L. Caragiale” (UNATC)}, \textcolor{magenta}{Bucharest, Romania}}

\cvitem{2019-02-08}{\textbf{Talk} at the Interactive Art, Science, and Technology in Western Canada (IAST Kelowna 2019) in conjunction with the \href{https://livingthingsfestival.com/}{Living Things 2019}, \textcolor{magenta}{Kelowna, BC, Canada}}

\cvitem{2018-06-01}{\textbf{Seminar}, \textit{Creative Artificial Intelligence for Music}, \href{http://www.miam.itu.edu.tr/}{Center for Advance Studies in Music (MIAM), Istanbul Technical University} \textcolor{magenta}{\.Istanbul, Turkey}}

\cvitem{2018-05-16}{\textbf{Seminar}, \textit{An Introduction to Metacreation and Musical Metacreation}, Music and Fine Arts Department, Middle East Technical University, \textcolor{magenta}{Ankara, Turkey}}

\cvitem{2018-02}{\textbf{Artist Talk} Modular approaches for Interactive Music Systems, at \textit{\href{https://www.facebook.com/events/2032925513589565}{Blue prints\_new prints}} at the Gold Saucer Studio in Vancouver, BC, Canada}

\cvitem{2017-09}{\textbf{Artist Talks} Three talks at the \href{https://www.aec.at/ai/en/iota}{Ars Electronica festival 2017 Artificial Intelligence}, \textcolor{magenta}{Linz, Austria}}

\cvitem{2017-06}{\textbf{Poster}, Tatar, K. \& Pasquier, P. (2017). MASOM: A Musical Agent Architecture based on Self-Organizing Maps, Affective Computing, and Variable Markov Models. Presented at the Eighth International Conference on Computational Creativity, ICCC 2017, \textcolor{magenta}{Atlanta, Georgia, USA}} 

\cvitem{2017}{\textbf{Demo}, \textit{Style Machine}, BC Tech Summit, Vancouver, BC, Canada}
\cvitem{2017}{\textbf{Demo}, \textit{Automatic Synthesizer Preset Generation with PresetGen}, BC Tech Summit, Vancouver, BC, Canada}

\cvitem{2016}{\textbf{Seminar}, \textit{An Introduction to Metacreation and Musical Metacreation}, Faculty of Computer and Informatics Engineering, Istanbul Technical University, \textcolor{magenta}{Istanbul, Turkey}}

\cvitem{2016}{\textbf{Workshop + Artist Talk}, \textit{Sound as a mediator in Interactive Arts: Challenges of Interdisciplinarity}, with Mirjana Prpa; at Vancouver ProMusica 2016, Vancouver, BC, Canada}

\cvitem{2016}{\textbf{Workshop + Artist Talk}, \textit{An interdisciplinary artwork: Pulse.Breath.Water}, with Mirjana Prpa; at \textit{movement.futures - May Residency 2016, School of Interactive Arts and Technology, Simon Fraser University, Vancouver, BC, Canada}}

\cvitem{2016}{\textbf{Workshop + Artist Talk}, \textit{The Potentials and Challenges  of VR, and Pulse.Breath.Water}, with Mirjana Prpa; at Chapel Sound Festival 2016, Vancouver, BC, Canada}

\cvitem{2016}{\textbf{Talk \& Demo} Prpa, M., Tatar, K., Pasquier, P., \& Riecke, B. E. (2016). \textit{Living In A Box: Potentials and Challenges of Existence in VR}, at \href{http://www.consumer-vr.com/}{the Consumer Virtual Reality (CVR) Conference, 2016}, Vancouver, BC, Canada}

\section{Research Utilization - Artworks}

\cvitem{2024-08-31}{\href{https://kivanctatar.com/Exposing-the-Bias-in-AI}{Exposing the Bias in Artificial Intelligence}. An audiovisual live performance at the \href{https://koami.art/}{Koami Arts Festival}, 300 in-person audience, Gothenburg, Sweden.}

\cvitem{2023-09}{\href{https://kivanctatar.com/expert-proc}{\textbf{Expert Procrastinator's Tool: Artificial Intelligence}}. A video artwork at the collective exhibiton \textit{Eskizden Piksele No.2}, with Nergiz   Yesil, Cem Sonel, Can Büyükberber, Afra Sönmez, Selçuk Artut, Alp Tugan, Güvenç Özel,   Candas Sisman; at the historical city of Troy,
as a part of the festival Troy Kültür Yolu \textcolor{magenta}{Troy, Canakkale, Turkey. Funded by Culture Ministry of Turkey,
received 18350 visitors in-person.}}

\cvitem{2023-08}{\href{https://kivanctatar.com/expert-proc}{\textbf{Expert Procrastinator's Tool: Artificial Intelligence}}. A video artwork at the collective exhibiton \textit{Eskizden Piksele No.2}, with Nergiz Yesil, Cem Sonel, Can Büyükberber,   Afra Sönmez, Selçuk Artut, Alp Tugan, Güvenç Özel, Candas Sisman; at the historical wine cellar in Kapadokya, as a part of the festival \textcolor{magenta}{Kapadokya} Kültür Yolu \textcolor{magenta}{Nevsehir, Turkey}. \textcolor{magenta}{Funded by Culture Ministry of Turkey, received 12670 visitors in-person.}}

\cvitem{2023-05}{\href{https://kivanctatar.com/expert-proc}{\textbf{Expert Procrastinator's Tool: Artificial Intelligence}}. An audiovisual performance at Chalmers Day 2023, received approx. 1500 in-person audience.}

\cvitem{2023-05-11}{\href{https://kivanctatar.com/Plastic-Biosphere}{\textbf{Plastic Biosphere No.2}} live audiovisual performance at Kunsthal Nord as a part of \href{https://www.communication.aau.dk/un-disciplinary-e73485#3rd--may}{\textit{Un-Disciplinary}} in Aalborg, Denmark.}
    
\cvitem{2022-10-23 2022-10-01}{\href{https://kivanctatar.com/Plastic-Biosphere}{\textbf{Plastic Biosphere No.3}}, a video artwork at the collective exhibiton ``Eskizden Piksele'', with Nergiz Yeşil, Cem Sonel, Can Büyükberber, Afra Sönmez, Selçuk Artut, Alp Tuğan, Güvenç Özel, Candaş Şişman; at the historical architecture ``Sığınak'', the shelter of the historical second parliament building of Turkey (which is currently the Atatürk Museum), as a part of the festival \textit{Başkent Kültür Yolu Ankara (Culture Road Ankara)}, received \textcolor{magenta}{27480 visitors} in-person in 23 days, \textcolor{magenta}{Ankara, Turkey}.}

\cvitem{2022-07-30}{\textbf{Coding the Latent No.2}, live coding performance using Deep Learning based audio generation with 3D audio and realtime visuals, as a part of \href{https://www.facebook.com/events/1235057850601595}{ double-bill: Tatar  | Avendaño \& Strauss} at \href{8EAST}{https://8east.ca} Vancouver, BC, Canada.}

\cvitem{2022-03-27 2022-03-25}{\href{https://kivanctatar.com/Plastic-Biosphere}{\textbf{Plastic Biosphere No.2}}, a live performance at a \href{https://www.facebook.com/events/4900673100021173/}{physical concert}, using AI for music and moving images, as a part of \href{https://4bidgallery.wordpress.com/2017/01/29/b-base/}{b.base expo} with Maria Pisiou at \href{https://4bidgallery.wordpress.com}{4bid Gallery} at OT301 in \textcolor{magenta}{Amsterdam, Netherlands}, invited through curatorial selection by 4bid gallery.}

\cvitem{2022-01-28}{\textbf{Coding the Latent No.1}, live coding performance using Deep Learning based audio generation with 3D audio and realtime visuals, as a part of \href{https://zkm.de/en/event/2022/01/on-the-fly-live-coding-hacklab}{Live Coding Hack Lab} of \href{https://onthefly.space}{on-the-fly}, at the Kubus, \textcolor{magenta}{\href{https://zkm.de/en/person/kivanc-tatar}{Center for New Media (ZKM)} Karlsruhe, Germany}}

\cvitem{2021-12}{\href{https://kivanctatar.com/Plastic-Biosphere}{\textbf{Plastic Biosphere No.2}}, a live performance stream, using AI for music and moving images, as a part of \href{https://www.sacmmt.com/}{Bowed Electrons 2021} hosted from \textcolor{magenta}{Cape Town, South Africa}, invited through curatorial selection by Theo Herbst from University of Cape Town.}

\cvitem{2021-09-23}{\href{https://kivanctatar.com/Plastic-Biosphere}{\textbf{Plastic Biosphere No.2}} as a part of collective concert \href{http://mutablesubject.ca/interplay_2021/}{interplay\_2021} with Sidi Chen, Find Mutya, Brandy Leary, Tamar Tabori, Raj Gill, Stéphanie Cyr, hosted from Vancouver, BC, Canada. Selected through an application process, peer-reviewed by three curators.}

\cvitem{2021-08-16}{\href{https://kivanctatar.com/Plastic-Biosphere}{\textbf{Plastic Biosphere No.2}} premier with \href{https://www.instagram.com/tamamamar}{Tamar Tabori}, a live performance stream using AI for music and moving images, with the support of \href{https://newmusic.org/}{Vancouver New Music}, supported by the grant (\hyperref[CCA9]{CCA-9}) that is acquired competitively.}

\cvitem{2021-07-15 2021-04-15}{\href{https://kivanctatar.com/gestalt-generation}{\textbf{gestalt generation no.1}} with \href{https://remysiu.com/}{Remy Siu}, as a part of collective exhibition \href{https://event.culture.tw/NTMOFA/portal/Registration/C0103MAction?useLanguage=en&actId=10050&request_locale=en}{\textit{Aethereal}}, at the \href{https://www.ntmofa.gov.tw/en/}{Taiwan National Museum of Fine Arts}, \textcolor{magenta}{Taichung, Taiwan}, supported by the grant (\hyperref[CCA7]{CCA-7}) that is acquired competitively.}

\cvitem{2021-01-03 2020-12-07}{\href{https://kivanctatar.com/Brush-of-AI}{\textbf{The Silhouettes of Istanbul No.3: The Brush of Artificial Intelligence}}, a video work showcased on 52 public large-scale screens scattered around Istanbul, as a part of Istanbul The Lights festival, produced by the \href{https://ciacef.org/}{Contemporary Istanbul Foundation (CIF)} and the \textcolor{magenta}{City of Istanbul, Turkey}, three NFTs from this series is available on \href{https://foundation.app/@kivanctatar}{Foundation}, invited through CIF curatorial selection.}

\cvitem{2020-09-09}{\textbf{Plastic Biosphere No.1} an interactive artwork using AI for music and moving images, exhibited as a part of the Kepler's Gardens Series, \textcolor{magenta}{Bucharest, Romania} Edition, at the \textcolor{magenta}{Ars Electronica 2020} in \textcolor{magenta}{Linz, Austria}, selected through an application process for the artist residency (\hyperref[AR3]{AR-3})}

\cvitem{2020-08-30}{\href{https://www.youtube.com/watch?v=tC9Cr1YDtHo}{\textbf{Instance}} telematic music-dance performance of Lucy Strauss and Sumalgy Nuro with the contributions of K{\i}van\c{c} Tatar, Bob Pritchard, Marina Thibeault, presented as online live stream.}

\cvitem{2020-06 2019-12}{\href{https://kivanctatar.com/Silhouettes-of-Istanbul}{\textbf{The Silhouettes of Istanbul}} In the collective exhibition \textit{Genetic Codes of Turkish Design, 4th Edition}, with Abra Design, Ahmet Rüstem Ekici,Ali Bakova, Alper Derinboğaz, Arife Design, Arzu Kaprol, Atasay, Atlas Harran, Atıl Kutoğlu, Aykut Erol, Başak Cankeş, Burcu Yıldız, Craft in İstanbul, Day Studio, Derya Geylani, Devrim Erbil, Dice Kayek , Dila Gökalp, Egemen Kemal Vuruşan, Ela Cindoruk, Elif Gönensay, Emin Barın, Faruk Malhan, Mehmet Girgiç, Fırat Neziroğlu, Geray Gencer, Rollic Games, Güvenç Özel, Hakan Sorar, Hamm Design, Hakan Akkaya, Hasan Kocam İrem Buğdaym İznik Çini Vakfı, Koleksiyon Kunter, Şekercioğlu, Melike Altınışık, Nazan Pak, Nohlab, Omar Baban, Özlem Tuna, Pınar Akkurt, \textcolor{magenta}{Refik Anadol}, Reo.tek, Toz Design, Under1Minute, Zehra Çobanlı, Zen Seramik... at the International Departures Lobby, \textcolor{magenta}{Istanbul Airport, Turkey}, invited through curatorial selection.}

\cvitem{2019-11-30}{\textbf{\href{https://kivanctatar.com/revive}{REVIVE}: \href{http://bocadellupo.com/revive-a-coda-guest-presentation/}{A CODA Guest Presentation}} [Performer \& Developer] concert with Philippe Pasquier and \href{remysiu.com}{Remy Siu} at the \href{http://bocadellupo.com/revive-a-coda-guest-presentation/}{Performance Works} (14 channel, 2 projections), co-produced by \href{http://bocadellupo.com/revive-a-coda-guest-presentation/}{Boca Del Lupo} Vancouver, BC, Canada.}

\cvitem{2019-10}{\href{https://kivanctatar.com/Machine-Learned-Landscape-01}{\textbf{Landscapes Past Futures}} [Artist \& Technologist] Prints and fixed-media video, in the exhibition as a part of the \textit{Index Media Arts Festival}, at the \textit{gnration} Gallery,  \textcolor{magenta}{Braga, Portugal}, invited through curatorial selection.}

\cvitem{2019-07-05}{\href{https://kivanctatar.com/u1m}{\textbf{Under 1 Minute (U1M):  Seismic Waves}} [Artist \& Technologist] concert at the \href{http://www.starofbosphorus.com.tr}{Star Bosphorus Data Center} with Mehmet \"Unal and Hakan Y{\i}lmaz, \textcolor{magenta}{\.Istanbul, Turkey}, invited through curatorial selection}

\cvitem{2019-09-12}{ \href{https://kivanctatar.com/u1m}{\textbf{Under 1 Minute (U1M): Digital Sculpture "The One"}} [Artist \& Technologist] exhibition at the \href{http://www.starofbosphorus.com.tr}{Contemporary Istanbul} PlugIn Exhibition, \textcolor{magenta}{74000 visitors in person, 40k social media shares and 10 million online views on social media within one week}, with Mehmet \"Unal and Hakan Y{\i}lmaz, \textcolor{magenta}{\.Istanbul, Turkey}, invited through CIF curatorial selection}

\cvitem{2019-07-18}{\href{https://kivanctatar.com/u1m}{\textbf{Under 1 Minute}} (U1M) [Artist \& Technologist] concert at the Harbiye Cemil Topuzlu Open-Air Theatre, the concert opening performance for Kenan Do\u{g}ulu (Turkish pop-star), approx. \textcolor{magenta}{10000 audience}, with Mehmet \"Unal and Hakan Y{\i}lmaz, \textcolor{magenta}{\.Istanbul, Turkey}}

\cvitem{2019-06-20}{\href{http://kivanctatar.com/revive}{\textbf{REVIVE}} [Performer \& Technologist] concert at \href{https://nycemf.org/}{New York Electroacoustic Music Festival (NYCEMF 2019)} with Philippe Pasquier and \href{remysiu.com}{Remy Siu} at the \href{https://www.fridmangallery.com/}{Fridman Gallery} (8 channel), \textcolor{magenta}{New York City, New York, USA}, selected through an application process and academic conference style peer-review process, supported by the grant (\hyperref[CCA3]{CCA-3}) that is acquired competitively.}

\cvitem{2019-04-03}{\href{https://kivanctatar.com/u1m}{\textbf{Under 1 Minute}} [Artist \& Technologist] concert at the \href{http://ameistanbul.com/events/ace-of-m-i-c-e-awards-ceremony/}{Ace of M.I.C.E Awards Ceremony}{ with Mehmet \"Unal and Hakan Y{\i}lmaz}, \textcolor{magenta}{\.Istanbul, Turkey}, invited through curatorial selection.}

\cvitem{2019-02-09}{\href{https://kivanctatar.com/revive}{\textbf{REVIVE}} [Performer \& Technologist] concert at the \href{https://livingthingsfestival.com/}{Living Things Festival 2019} with Philippe Pasquier, at Kelowna Art Gallery (6 speakers), \textcolor{magenta}{Kelowna, BC, Canada}, invited through curatorial selection.}

\cvitem{2018-12-14}{\href{https://kivanctatar.com/zeta}{\textbf{ZETA}} [Artist \& Technologist] touch-based interactive installation for 360 immersive interface with multichannel audio, at the \href{http://immersivelab.zhdk.ch/}{Immersive Lab} at \href{https://www.zhdk.ch/en/research/icst}{the Institute for Computer Music and Sound Technologies}, \href{https://www.zhdk.ch/}{Zurich University of the Arts} with \href{https://philipepasquier.com}{Philippe Pasquier}; \textcolor{magenta}{Z\"urich, Switzerland}, invited through curatorial selection, supported by the grant (\hyperref[CCA2]{CCA-2}) that is acquired competitively.}

\cvitem{2018-12-08}{\href{https://kivanctatar.com/revive}{\textbf{REVIVE}} [Performer \& Technologist] concert at \href{https://www.zhdk.ch/veranstaltung/37286}{Zurich University of the Arts} with Philippe Pasquier and \href{http://remysiu.com}{Remy Siu} at the Konzertsaal 2 (26 speakers and 4 subwoofers), \textcolor{magenta}{Z\"urich, Switzerland}}

\cvitem{2018-08-23}{\href{https://kivanctatar.com/revive}{\textbf{REVIVE}} [Performer \& Technologist] concert at \href{https://www.mutek.org/en/archives/events/2018/1531-satosphere-3-revive}{\textcolor{magenta}{MUTEK Montreal 2018}} with Philippe Pasquier and \href{http://remysiu.com}{Remy Siu} at the Soci\`et\`e des Arts Technologique (SAT) dome (157 speakers and 8 projectors), 2 performances, \textcolor{magenta}{Montreal, Quebec, Canada}, invited through curatorial selection, supported by the grant (\hyperref[CCA2]{CCA-2}) that is acquired competitively.}

\cvitem{2018-08-31}{\href{http://kivanctatar.com/respire}{\textbf{RESPIRE}} [Artist \& Technologist] exhibition at \href{http://cinevolutionmedia.com/portfolio-item/dc2018-air/}{Digital Carnival 2018} with Mirjana Prpa and Philippe Pasquier Vancouver, BC, Canada, invited through curatorial selection.}

\cvitem{2018-05-20}{\textbf{In the Engine} [Composer] in the collective concert, \href{https://www.facebook.com/events/s/istanbul-soundscape-project-ha/139446233576758/?ti=icl}{\.{I}stanbul Soundspace Project: Haydarpa\c{s}a'da Bir Gar.} @\href{http://www.arkaoda.com/}{arkaoda}, \textcolor{magenta}{\.{I}stanbul, Turkey}, invited through curatorial selection.}

\cvitem{2018-04-24}{\href{https://kivanctatar.com/revive}{\textbf{REVIVE}} [Performer\&Developer] concert at \href{http://sat.qc.ca/fr/connexions}{CHI Connexitions} with Philippe Pasquier and \href{http://remysiu.com}{Remy Siu} at the Soci\`et\`e des Arts Technologique (SAT) dome (157 speakers and 8 projectors), 7 performances as a part of \href{https://chi2018.acm.org/}{\textcolor{magenta}{CHI 2018}} Conference on Human Factors in Computing Systems, \textcolor{magenta}{Montreal, Quebec, Canada}}

\cvitem{2018-04-27}{\href{http://kivanctatar.com/respire}{\textbf{RESPIRE}} [Artist \& Technologist] exhibition at \href{https://chi2018.acm.org/}{CHI Virtual Reality exhibiton} with Mirjana Prpa and Philippe Pasquier, as a part of \href{https://chi2018.acm.org/}{\textcolor{magenta}{CHI 2018}} Conference on Human Factors in Computing Systems, \textcolor{magenta}{Montreal, Quebec, Canada}.}

\cvitem{2018-04-18}{\href{http://kivanctatar.com/respire}{\textbf{RESPIRE}} [Artist \& Technologist] exhibition at \href{http://vanartgallery.bc.ca/}{\textcolor{magenta}{Vancouver Art Gallery}} with  Mirjana Prpa and Philippe Pasquier as a part of \href{http://mwx2018.org/}{the conference Museums and the Web MWX18}, Vancouver, BC, Canada}

\cvitem{2018-03-08}{\textbf{Eternal Pink Noise Machine} [Artist \& Technologist], sound installation with Philippe Pasquier, at the \textit{\href{https://www.facebook.com/events/217009415527752}{Pink Noise Pop Up}} Exhibition in \textcolor{magenta}{Seoul, South Korea.}}

\cvitem{2018-02-04}{\textbf{Trumpet \& Electronics} Solo performance at the collective concert \href{https://www.facebook.com/events/2032925513589565/}{\textit{Blue prints\_new prints}} with Alanna Ho, Ben Brown, and Roxanne Nesbitt; at the Gold Saucer Studio, Vancouver, Canada}

%\cvitem{2018::2017}{\textbf{Collaborator \& Developer} \href{http://lindabouchard.com/}{Linda Bouchard}'s Live Structures project}

\cvitem{2017-11-16}{\href{https://www.nowsociety.org/event/trading-places-un\%C3\%A9change-dimprovisateurs-montr\%C3\%A9al-vancouver}{\textbf{Trading Places: Un \'{E}change d'Improvisateurs}} Concert [Trumpet\&Electronics]; with Vicky Mettler, Torsten Muller, and Ross Birdwise; at the Roundhouse, Vancouver, Canada}

\cvitem{2017-09-24}{\textbf{Theta} [Artist \& Technologist] MASOM joins two media art companies from Istanbul, \href{http://ouchhh.tv/}{Ouchhh} and \href{http://audiofil.io/}{AudioFil} for a projection mapping piece on \textcolor{magenta}{the Facade of the Bolshoi Theatre}, at the \href{http://lightfest.ru/en/}{Circle of Light 2017}, \textcolor{magenta}{Moscow, Russia}}

\cvitem{2017-09-16}{\textbf{Theta} [Artist \& Technologist] MASOM joins two media art companies from Istanbul, \href{http://ouchhh.tv/}{Ouchhh} and \href{http://audiofil.io/}{AudioFil} for \href{http://www.imapp.ro/2017-2/}{a projection mapping piece} on \textcolor{magenta}{the Facade of the Romanian Parliament}, at the \href{http://imapp.ro/}{IMapp Bucharest 2017}, \textcolor{magenta}{Bucharest, Romania}}

\cvitem{2017-09-07}{\href{https://www.facebook.com/Ouchhh.tv/videos/1492873787455600/}{\textbf{IOTA\_AI}} [Artist \& Technologist], MASOM joins two media art companies from Istanbul, \href{http://ouchhh.tv/}{Ouchhh} and \href{http://audiofil.io/}{AudioFil} for a performance at the \href{https://www.aec.at/ai/en/iota}{\textcolor{magenta}{Ars Electronica Festival 2017}} with the theme Artificial Intelligence. The team performed three times at the Deep Space 8K during the festival. \textcolor{magenta}{Linz, Austria}}

\cvitem{2017-06-19}{\textbf{MA\_Test SOM\_Pattern} [Performer \& Technologist] with the project \href{https://kivanctatar.wordpress.com/patar}{Patar}, in the collective concert by \href{https://musicalmetacreation.org/mume-2017-concert}{\textit{Musical Metacreation Concert}} \textcolor{magenta}{Atlanta, Georgia, USA}}

\cvitem{2017-04-22}{\textbf{Patar @CoCreaTive} [Performer \& Technologist] in the collective concert \textit{Barely Constrained} by \href{https://cocreative.wordpress.com/}{Co.Crea.Tive} Vancouver, BC, Canada}

\cvitem{2017-03-23}{\href{https://cocreative.wordpress.com/2017/04/30/a-big-masom-family/}{\textbf{A Big MASOM Family}} [Performer \& Technologist] in the collective concert \textit{RE-UN-SOLVED} by \href{https://cocreative.wordpress.com/}{Co.Crea.Tive} Vancouver, BC, Canada}

\cvitem{2016-12-25}{\textbf{Tatar and MASOM take the AID train} [Performer \& Technologist] in the collective concert \href{https://www.facebook.com/events/1769695369957515/}{\textit{Take the AID Train}} by \href{http://artisdead.in}{A.I.D}, \textcolor{magenta}{\.{I}stanbul, Turkey}}

\cvitem{2016-12-02}{\href{https://www.nowsociety.org/madmethod-december-2-3}{\textbf{madMethod}} [Performer \& Technologist] by NOW Society, with Stefan Smulovitz, Sammy Chien, MASOM, Philippe Pasquier, Lisa Cay Miller, Jon Bentley, JP Carter, James Meger, Skye Brooks, at Orpheum Annex, Vancouver, BC, Canada}

\cvitem{2016-11-10} {\href{https://kivanctatar.com/Pulse-Breath-Water}{\textbf{Pulse.Breath.Water}} [Artist \& Technologist] with Mirjana Prpa, Philippe Pasquier, Bernhard Reicke in the VR exhibition at \textcolor{magenta}{\href{http://www.mutek.org/en/img/2016/artworks}{MUTEK\_IMG}} - Virtual Reality (head mounted display and headphones), generative audio, embodied interaction (via breath sensors), \textcolor{magenta}{Montreal, Quebec, Canada}}

\cvitem{2016-10-22}{\href{https://kivanctatar.com/MASOM-0-01}{\textbf{A Conversation with AI}} [Performer \& Technologist] in the collective concert \textit{Open to Enter} by \href{https://cocreative.wordpress.com/}{CoCreaTive}, Vancouver, BC, Canada}

\cvitem{2016-08-31}{\href{https://smc2016.hfmt-hamburg.de/?session=musebot-chill-out-session-a-continuously-running-installation}{\textbf{Musebot Chill-out Session}} [Artist \& Technologist] with Arne Eigenfeldt, Paul Paroczai, Oliver Bown, Ben Carey, Toby Gifford, Jeffrey Morris, and Si Wait, Sound and Music Computing, SMC 2016, \textcolor{magenta}{Hamburg, Germany}}

\cvitem{2016-07-18} {\textbf{\href{https://kivanctatar.com/POEMA}{P.O.E.M.A.}} [Artist \& Technologist] with Regina Miranda, Mirjana Prpa, Philippe Pasquier, and Bernhard Reicke; Generative Audio (quadrophonic setup), Choreographic Installation, Virtual Reality (head mounted display and projection), Embodied Interaction (via respiration sensors), at the gallery \textit{Oi Futuro}, as a part of the cultural program at \textcolor{magenta}{OLYMPICS 2016, Rio de Janeiro, Brazil}}

\cvitem{2016-07-11}{\textbf{Musebot Chill-out Session} [Artist \& Technologist] sound installation with Arne Eigenfeldt, Paul Paroczai, Oliver Bown, Ben Carey, Toby Gifford, and Jeffrey Morris, International Conference on New Interfaces for Musical Expression, NIME 2016, \textcolor{magenta}{Brisbane, Australia}}

\cvitem{2016-04-07} {\href{https://kivanctatar.com/Organic-Strategies}{\textbf{Organic Strategies}} [Trumpet \& Electronics] Matthew Ariaratnam in the collective concert: \textit{Constrained Improv}, @\textit{Red Gate Arts Society}, Vancouver, BC, Canada}

\cvitem{2016-03-10} {\href{https://kivanctatar.com/Pulse-Breath-Water}{\textbf{Pulse.Breath.Water}} [Artist \& Technologist] with Mirjana Prpa, Philippe Pasquier, and Bernhard Reicke, in the exhibition \href{http://oneartspace.com/2016/03/10/scorestraces-exposing-the-body-through-computation/}{\textit{Scores+Traces: exposing the body through computation}} - Virtual Reality (head mounted display and headphones), generative audio, embodied interaction (via breath sensors) @\textit{One Art Space}, \textcolor{magenta}{New York, NY, USA}}

\cvitem{2016-03-10} {\textbf{Tuned Ocean no.2} [Artist \& Technologist] sound installation in the exhibition \textit{Scores+Traces: exposing the body through computation}  - sound installation, generative audio, @\textit{One Art Space}, \textcolor{magenta}{New York, NY, USA}.}

\cvitem{2016-01-23}{\href{https://kivanctatar.com/Code-of-Silence-2}{\textbf{Code of Silence Nb.2}} [Composer] graphic notation for any number of performers. Premiered by Plastic Acid Orchestra at One-Page Score event. Vancouver, BC, Canada}

\cvitem{2016-01-16}{\href{https://kivanctatar.com/Code-of-Silence}{\textbf{Code of Silence}} [Trumpet \& Electronics] in the collective concert - \textit{Improvised Resolutions} @\textit{Gold Saucer Studio}, Vancouver, BC, Canada}

\cvitem{2015-12-09}{\textbf{Musebot Chill-out Session} [Artist \& Technologist] with Arne Eigenfeldt, Paul Paroczai, Oliver Bown, Ben Carey, Toby Gifford, and Jeffrey Morris, Generative Art Conference 2016, \textcolor{magenta}{Venice, Italy}}

\cvitem{2015-11-16}{\href{https://kivanctatar.com/Musebots-for-PROCJAM-2015}{\textbf{Musebots for PROCJAM 2015}} [Artist \& Technologist] generative music piece with Arne Eigenfeldt, Oliver Bown, Ben Carey, Toby Gifford}

\cvitem{2015-06-07}{\href{https://threelittlereddots.org/performances/together-apart/}{\textbf{Together () Apart}} [Sound Designer] performance piece by \href{https://threelittlereddots.org/}{Isabelle Kirouac}, Vancouver, BC, Canada}

\cvitem{2015-08}{\href{https://kivanctatar.com/Black-and-white}{\textbf{Black and White: Where the bomb meets the toys}} [Composer] graphic notation for three performers, Vancouver, BC, Canada}

\cvitem{2015-08-22}{\href{https://vinesartfestival.com/wp-content/uploads/2018/05/2015_festival_poster.jpg}{\textbf{Dissonant Disco Collective}} performance with made instruments and trumpet at \textit{Vines Festival}, Vancouver, BC, Canada}

\cvitem{2015-04-23}{\href{https://www.facebook.com/events/sfu-school-for-the-contemporary-arts/antinomial-antiphonies-part-of-the-mixed-greens-performance-series/468911126593214/}{\textbf{Antiphons}} [Performer] composition by Ben Wylie \textit{Antinomial Antiphonies, Mixed Greens Performance Series}, SFU Woodwards, Vancouver, BC, Canada}

\cvitem{2015-03-31}{\href{https://kivanctatar.com/Deep-Breath}{\textbf{Deep Breath}} [Trumpet \& Electronics] solo live performance, Black Box, Interactive Arts and Technology, SFU, Surrey, BC Canada}

\cvitem{2014-05-19}{\href{https://kivanctatar.com/Tat-Kal-Dem}{\textbf{Tat-Kal-Dem trio}} [Trumpet \& Electronics] Karakedi, \.{I}stanbul, Turkey}

\cvitem{2014-05-05}{\href{https://kivanctatar.com/Sonic-Arts-Day}{\textbf{Soundscapes from Poland}} [Trumpet] Sonic Arts Day Concert, free improvisation session, Mustafa Kemal Hall, \.{I}stanbul}

\cvitem{2013-05-02}{\textbf{Tuned Ocean} for electronics and recorded piano, live performance, \href{https://www.facebook.com/events/140813849439513}{ELECTROSONIC CITY 3.0}, Borusan Music House, \.{I}stanbul}

\cvitem{2013-03-16}{\textbf{Tin Men and the Telephone (NL) and Furt(DE, UK):} Do\u{g}a\c{c}lamada Avangard Perspektifler [Guest Musician, Trumpet \& Electronics]}

\cvitem{2013-02-09}{\textbf{Take it} [Electronics] MIAM Groove Collective Concert, Wake Up Call, \.{I}stanbul}

\cvitem{2012-12-05}{\href{https://www.facebook.com/events/565785350101463/}{\textbf{ Beyond Trumpet}} [Trumpet \& Electronics] live solo performance, four channels, MIAM NOISE COLLECTIVE XV Concert, MIAM Recording Studio, \.{I}stanbul}

\cvitem{2012-10-31}{\textbf{\.{I}stanbul in Boring Stereo without Clarinet} [Composer] four channels electroacoustic piece, MIAM Electroacoustic Collective XV Concert, MIAM Recording Studio, \.{I}stanbul}

%\cvitem{2012}{\textbf{A Jazz Installation} [Composer] performance with Sound Installation, METU Architecture Department Building, Ankara}

%\cvitem{2011}{\textbf{Huzur} [Composer] a sound installation with participatory interaction, METU Library Exhibition Hall, ANKARA}

\cvitem{2008-2012}{\textbf{Principal Trumpet}, METU Big Band, two concerts per year, Ankara}

\cvitem{2009-09-12}{\textbf{Point, Line, Space and Sound} [Trumpet \& Electronics] performance as a part of Bauhaus project, in collaboration between Bauhaus University and Middle East Technical University \textcolor{magenta}{Weimar, Germany}}

\cvitem{2009-07-30}{\textbf{Point, Line, Space and Sound} [Trumpet \& Electronics] performance as a part of Bauhaus project, in collaboration between Bauhaus University and Middle East Technical University, METU, Ankara, Turkey}



\section{Research Utilization - Press}

\cvitem{2023-11}{I appeared on a radio show in Sweden, on FM 103.1, called Discokaputt, and presented my musical AI works. This was a result of engagements with a local cultural organization (non-profit) called Skogen.}

\cvitem{2023}{Three exhibitions in Turkey, in Ankara, Capadocia (Nev\.sehir), Troy (\.Canakkale) have been widely covered on Turkish National television several times, in \href{https://www.trt2.com.tr/}{TRT 2} and other TV channels, as a part of the \textit{Kultur Yolu} event series funded by the Ministry of Culture of Turkey.}

\cvitem{2020-05-22}{\textbf{Neural} - \href{http://neural.it/2020/05/respire-breathing-in-sound-and-vision/}{\textit{Respire, breathing in sound and vision}}}

\cvitem{2019-12-12}{\textbf{Exclaim!} - \href{https://exclaim.ca/music/article/the_artificial_intelligence_takeover}{\textit{The Artificial Intelligence Takeover of Music in 2019}}, with Holly Herndon, YACHT, Endel, and Algorave artists.}
\cvitem{2019-12-10}{\textbf{The La Source}, Volume 20, Issue 09 -  \href{http://thelasource.com/en/2019/11/18/kivanc-tatar-crossing-the-boundaries-of-science-and-the-arts/}{\textit{Kıvanç Tatar: crossing the boundaries of science and the arts}}, authored by Xi Chen.}
\cvitem{2019-10-16}{\textbf{SFU News} - \href{https://www.sfu.ca/sfunews/stories/2019/10/graduate-takes-new-media-to-new-creative-levels-with-artificial-.html?fbclid=IwAR3gZqtUby9aNQUr0-AiWr_47nLJ1tXY3LGFCLIbE297u-OQYArU0o5qM78}{\textit{Graduate takes New Media to new creative levels with Artificial Intelligence}}, Vancouver, BC, Canada}

\cvitem{2020-06-19}{\textbf{SFU News} - \href{https://www.sfu.ca/siat/stories/research/exploring-creative-artificial-intelligence.html}{\textit{Exploring Creative Artificial Intelligence}}, Vancouver, BC, Canada}

\cvitem{2019-01-01}{\href{https://kivanctatar.com/2019-istanbul-art-news}{\textbf{Istanbul Arts News}} - Piyasa - January Issue, Interview on Creative AI, Turkey, authored by G\"uniz An\i l}

\cvitem{2018-01-04}{\textbf{Artful Living}, \href{https://www.artfulliving.com.tr/sanat/dev-kadro-iki-gorsel-sanatci-bir-besteci-ve-bir-yapay-zek-i-14437}{\textit{Dev Kadro: \.Iki G\"orsel Sanat\c{c}{\i}, Bir Besteci ve Bir Yapay Zeka}}, authored by Esra \"Ozkan, Turkey}

\cvitem{2016-11-06}{\textbf{VANDOCUMENT}, \href{https://vandocument.com/2016/11/a-conversation-with-artificial-intelligence/}{\textit{A Conversation with Artificial Intelligence}}, authored by Ash Tanasiychuk, Vancouver, BC, Canada}

\cvitem{2016-08-08}{\textbf{MetroNews Vancouver}, local newspaper front page, \textit{Vancouver artists making waves at the Olympics}, Vancouver, BC, Canada}

\cvitem{2016-08-09}{\textbf{Daily Hive}, \href{http://dailyhive.com/vancouver/rio-olympics-sfu-art-installation}{\textit{Rio Olympics showcases SFU virtual reality dance installation}}, Vancouver, BC, Canada}

\cvitem{2016-08-17}{\href{https://kivanctatar.com/POEMA-SFU-news}{\textbf{SFU News}, \textit{SIAT art project at Rio Olympics takes your breath away}}, authored by Allen Tung, Vancouver, BC, Canada}

\cvitem{2016-07-24}{\textbf{O Imparcial}, \href{https://oimparcial.com.br/entretenimento-e-cultura/2016/07/performance-imersiva-onde-o-publico-experimenta-a-realidade-virtual/}{\textit{Imersiva onde o publico experimenta a realidade virtual}, authored by Camila Pereira}, \textcolor{magenta}{Brazil}}

\cvitem{2016-07-24}{\href{https://www.acritica.com/channels/entretenimento/news/companhia-de-regina-miranda-mistura-danca-e-realidade-virtual-em-obra-que-esta-em-cartaz-no-rj}{\textbf{A Critica}, \textit{Companhia de Regina Miranda mistura danca e realidade virtual em obra que esta em cartaz no rj}, authored by Rosiel Mendon\c{c}a, \textcolor{magenta}{Brazil}}}

\cvitem{2016-06-22}{\textbf{Glamurama} ,\href{https://glamurama.uol.com.br/regina-miranda-prepara-danca-instalacao-para-ingles-ver-nas-olimpiadas/}{\textit{Regina Miranda prepara dança instalação para inglês ver na Olimpíada}}, \textcolor{magenta}{Brazil}}

\section{Awards}
\cvitem{2019}{Travel \& Minor Research Award, SFU - CAD \$1000}
\cvitem{2019}{Graduate Fellowship Award, SFU - CAD \$3250}
\cvitem{2018}{FCAT Graduate Fellowship Award, SFU - CAD \$3250}
\cvitem{2018}{Graduate Fellowship Award, SFU - CAD \$6500}
\cvitem{2018}{President's PhD Scholarship, SFU - CAD \$6500}
\cvitem{2018}{Graduate Fellowship Award, SFU - CAD \$6500}
\cvitem{2017}{\href{http://lumenprize.com/page/introducing-2017-shortlist}{Lumen Prize} - Shortlisted in the category of Artificial Intelligence}
\cvitem{2017}{\href{http://www.human-competitive.org/awards}{14th Annual ``Humies'' Awards - 2017} - Finalist}
\cvitem{2017}{Travel \& Minor Research Award, SFU - CAD \$500}
\cvitem{2017}{Graduate Fellowship Award, SFU - CAD \$6500}
\cvitem{2016}{Graduate Fellowship Award, SFU - CAD \$6500}
\cvitem{2016}{Travel \& Minor Research Award (TMRA), SIAT, SFU - CAD \$900}
\cvitem{2014}{Graduate Fellowship Entrance Award, SFU - CAD \$6250}

\section{Technical Skills}
\cvitem{}{\small The related publications are mentioned in parenthesis.}
\cvitem{Coding}{Python (J1, J5, C2, C6, C8, AX1), Javascript  (J6, C3), C++ (C3), GLSL (C8, C6), Java (TH1), Matlab (C3), PureData (J1), Cycling74's Max (J6, J5, AX1, J5, J3, J1, C9, C8, C6, C5, C4, C3, C1), Processing, Arduino, Touch Designer (J6, C6, C8)}

\cvitem{Digital Technology}{Signal Processing (all pubs), Digital Signal Processing (all pubs), Audio Programming (all pubs), Deep Learning - Pytorch, Tensorflow (C14, C12, J5, AX1), Machine Learning (all pubs except J4), Pattern Matching and Recognition (all pubs except J4), Evolutionary Computation (J1 - Distributed Evolutionary Algorithms in Python - DEAP), Parallel Computation (J6, J1, AX1), Cloud Computing - Compute Canada, \href{https://www.xsede.org/}{XSEDE}, \href{https://www.ibm.com/watson}{IBM Watson} and \href{https://www.ibm.com/cloud/bare-metal-servers}{IBM Bare Metal} (C14, J6, J1, AX1), Multi-agent Systems (J2, J3, J4, J5, J6, C1, C3, C4, C5, C6, C8, C9)}
\cvitem{Electronics}{Analog and Digital Circuitry, Logic Design, Signals and Systems (all pubs), Feedback Systems, Discrete Time Control Systems (J6, J5, J3, C6, C8, C9), Microprocessor Programming, Printed Circuit Board Design, Audio Systems (all pubs)}

\section{Software Skills}
\cvitem{Electronics}{LTSPICE, Multisim}
\cvitem{Statistics}{SPSS (C7)}
\cvitem{Music Production}{Pro Tools, Ableton Live, Reaper, Cubase, Audacity, Shotcut, Waves \& FabFilter Plugins}
    
\section{Music Releases}
\cvitem{2021}{Plastic Biosphere No.3, Auto Impulse, Independent}
\cvitem{2017}{\textbf{OP-1}, Beta test: Opposition is overrated, \c{C}EK\.{I}\c{C}, Independent}
\cvitem{2016}{\textbf{Electronics}, Self Distruption, K{\i}van\c{c} Tatar, Independent}
\cvitem{2016}{\textbf{Trumpet\&Electronics}, Live at Red Gate, Organic Strategies, Independent}
\cvitem{2014}{\textbf{Composer}, Early Works (2012-2014), K{\i}van\c{c} Tatar, Independent}
\cvitem{2012}{\textbf{Composer}, Dialectic, K{\i}van\c{c} Tatar, Independent}
\cvitem{2011}{\textbf{Composer}, Huzur, K{\i}van\c{c} Tatar, Independent}



\section{Other - Research Leadership Training}

\cvitem{2021-2026}{Participating in WASP-HS research leardership program.}
\cvitem{2023-04}{Completed Chalmers Research Leardership program.}

\section{Other - Residencies}
\cvitem{2022-01}{{\textbf{Artist in Resident (AR-4):} at \href{https://zkm.de/en}{Center for Art and Media Karlsruhe}, as part of \href{https://onthefly.space/read/on-the-fly-live-coding-research-open-call-results}{on-the-fly} project sponsored by Creative Europe Programme of European Union, selected as an artist in resident out of 76 applications, in \textcolor{magenta}{Karlsruhe, Germany}}}
\phantomsection
\label{AR3}
\cvitem{2020-07::10}{\textbf{Artist in Resident (AR-3):} Virtual artist residency at \href{https://cinetic.arts.ro/evenimente/rezidentele-cinetic-2020/}{the International Center for Research and Education in Innovative and Creative Technologies (CINETic), National University of Theatre and Film “I.L. Caragiale” Bucharest (UNATC)}, \textcolor{magenta}{Bucharest, Romania}}	
\cvitem{2019-03}{\textbf{Visiting Researcher:} the  \href{https://www.hf.uio.no/ritmo/english/}{the Centre for Interdisciplinary Studies in Rhythm, Time and Motion (RITMO) at the University of Oslo} \textcolor{magenta}{Oslo, Norway}}
\cvitem{2018-12}{\textbf{Artist in Resident:} Institute for Computer Music and Sound Technologies (ICST), Zurich University of Arts (ZHdK), \textcolor{magenta}{Zurich, Switzerland}}
\cvitem{05/2017}{\textbf{Visiting Researcher:}\href{https://www.cirmmt.org/}{ the Centre for Interdisciplinary Research in Music Media and Technology (CIRMMT)}, at \href{https://www.mcgill.ca/music/}{the Schulich School of Music at McGill University} \textcolor{magenta}{Montreal, QC, Canada}}
\cvitem{2016-05}{\textbf{Visiting Researcher: }\href{http://movingstories.ca/events/}{\textit{movement.futures} - May Residency 2016} at the Emily Carl University of Arts,  Vancouver, BC, Canada}
\cvitem{2009-07::09}{\textbf{Artist in Resident:}\textit{ Bauhaus Project}, Bauhaus University in collaboration with the Middle East Tehnical University, \textcolor{magenta}{Weimar, Germany} and Ankara, Turkey}

\section{Academic Publications}
	
	\cvitem{}{\begin{center}\small\textcolor{magenta}{\textcolor{magenta}{Journal Papers}}\end{center}}

    \cvitem{J8}{Cotton, K., Kaila, A.K.,Jääskeläinen, P., Holzapel, A., \textcolor{magenta}{Tatar, K}. Imploding between the Facts and Concerns of Musical-AI: A Multidisciplinary Method for Analysing Human-AI Musical Interaction. Accepted with minor revisions to Humanities and Social Sciences Communications, Nature.}

    \cvitem{J7}{*\textcolor{magenta}{Tatar, K.}, Ericson P., Cotton K., Núñez del Prado P. T., Batlle-Roca R., Cabrero-Daniel, B, Ljungblad S., Diapoulis G., Hussain J. (2024) \textit{A Shift In Artistic Practices through Artificial Intelligence}.  Leonardo Journal, MIT Press.}
 
	\cvitem{J6}{Strauss, L., \textcolor{magenta}{Tatar, K.}, Nuro, S. (2021). Iterative Design Processes and Soma-based Practices in \textit{instance}. Organised Sound, Special Issue on "Collective and Networked Sound Practices.''. Cambridge Press. DOI: 10.1162\textbackslash leon\_a\_02523. \href{https://arxiv.org/pdf/2306.10054}{Open access pdf}}
		
	\cvitem{J5}{\textcolor{magenta}{Tatar, K.}, Bisig, D., \& Pasquier, P. Latent Timbre Synthesis: Audio-based variational auto-encoders for music composition and sound design applications. The Special Issue of \href{https://www.springer.com/journal/521}{Neural Computing and Applications}: ``Networks in Art, Sound and Design.'' Springer. \url{https://doi.org/10.1007/s00521-020-05424-2}}
	
	\cvitem{J4}{**\textcolor{magenta}{Tatar, K.}. (2020). Initial Remarks on Analyzing Acousmatic Music from the Perspective of Multi-agents. Array, special issue on Agency. International Computer Music Association. http://dx.doi.org/10.25532/OPARA-46}
	
	\cvitem{J3}{*\textcolor{magenta}{Tatar, K.}, Prpa M., \& Pasquier, P. (2019). A Virtual Reality Art Piece with a Musical Agent guided by Respiratory Interaction: \textit{Respire}. \textit{Leonardo Music Journal} Vol. 29, p. 19--24. \textcolor{magenta}{Cover of the issue}. \url{https://doi.org/10.1162/lmj_a_01057} }
	
	\cvitem{J2}{\textcolor{magenta}{Tatar, K.}, \& Pasquier, P. (2018). Musical agents: A typology and state of the art towards Musical Metacreation. \textit{Journal of New Music Research}, 48(1), 56–-105. \url{https://doi.org/10.1080/09298215.2018.1511736}.}
	
	\cvitem{J1}{\textcolor{magenta}{Tatar, K.}, Macret, M., \& Pasquier, P. (2016). Automatic Synthesizer Preset Generation with PresetGen. \textit{Journal of New Music Research}, 45(2), 124--144. \url{http://doi.org/10.1080/09298215.2016.1175481}}
	
	\cvitem{}{\begin{center}\small\textcolor{magenta}{\textcolor{magenta}{Conference Papers}}\end{center}}

    

    \cvitem{C19}{Chen. J., \textcolor{magenta}{Tatar K.}, Zappi, V. \textit{A Deep Learning Framework for Musical Acoustics Simulations}. In proceedings of AI Music Creativity 2024. \url{https://aimc2024.pubpub.org/pub/5cl1cvmy/release/1}}

    \cvitem{C18}{Cotton. K., \textcolor{magenta}{Tatar K.}. \textit{Sounding out extra-normal AI voice: Non-normative musical engagements with normative AI voice and speech technologies}. In proceedings of AI Music Creativity 2024. \url{https://aimc2024.pubpub.org/pub/extranormal-aivoice/release/1}}

    \cvitem{C17}{Caravati. M, \textcolor{magenta}{Tatar K.}. \textit{Interfacing ErgoJr with Creative Coding Platforms}. In proceedings of the 9th International ACM Conference on Movement and Computing.\url{https://doi.org/10.1145/3658852.3659082}}

    \cvitem{C16}{Cotton. K.,de Vries, Katja, and \textcolor{magenta}{Tatar K.}. \textit{Singing for the Missing: Bringing the Body Back to AI Voice and Speech Technologies}. In proceedings of the 9th International ACM Conference on Movement and Computing. \url{https://doi.org/10.1145/3658852.3659065}}

    \cvitem{C15}{Cotton. K.,\textcolor{magenta}{Tatar K.},\textit{ Caring Trouble and Musical AI: Considerations towards a Feminist Musical AI}. In Proceedings of AI Music Creativity Conference 2023. Received the highest ranking (5/5) from all three reviewers. \url{https://aimc2023.pubpub.org/pub/zwjy371l}}

    \cvitem{C14}{\textcolor{magenta}{Tatar K.},  Cotton, K., Bisig, D. (2023). \textit{Sound Design Strategies for Latent Audio Space Explorations Using Deep Learning Architectures}. In Proceedings of Sound and Music Computing 2023.}
 
	\cvitem{C13}{Obaid M., \textcolor{magenta}{Tatar K.}, Wiberg M., Said A., Rost M., Weilenmann A., Johal W., Eyssel F. (2022). \textit{Social Drones for Health and Well-being}. In Adjunct Proceedings of the 2022 Nordic Human-Computer Interaction (Nordic-CHI 2022) Conference.}
 
	\cvitem{C12}{Zappi V,. \textcolor{magenta}{Tatar, K.} (2022) \href{https://embedded-ai-for-nime.github.io/}{\textit{The Neuralacoustics Project: Exploring Deep-Learning for Lightweight Numerical Modeling Synthesis}}. Victor Zappi and K{\i}van\c{c} Tatar. In \textit{Embedded AI for NIME: Challenges and Opportunities} Workshop at New Interfaces for Musical Expression (NIME) 2022.}
	
    \cvitem{C11}{Diapoulis G., Zannos I., \textcolor{magenta}{Tatar, K.}, Dahlstedt P. (2022). Bottom-up live coding: Analysis of continuous interactions towards predicting programming behaviours. In Proceedings of New Interfaces for Music Expression Conference 2022.}
	
	\cvitem{C10}{Bisig D., \textcolor{magenta}{Tatar, K.} (2021). Raw Music from Free Movements: Early Experiments in Using Machine Learning to Create Raw Audio from Dance Movements. Accepted to AI Music Creativity Conference 2021 (AIMC 2021).}
	
	\cvitem{C9}{Boerson R., Liu-Rosenbaum A., \textcolor{magenta}{Tatar K.}, Pasquier P. (2020). Chatterbox: an interactive system of gibberish agents. Accepted to the International Symposium of Electronic Arts (ISEA 2020)}
	
	\cvitem{C8}{\textcolor{magenta}{Tatar K.}, Pasquier P., Siu R. (2019). Audio-based Musical Artificial Intelligence and Audio-Reactive Visual Agents in Revive. In Proceedings of the International Computer Music Conference and New York City Electroacoustic Music Festival 2019 (ICMC-NYCEMF 2019)}
	
	\cvitem{C7}{Fan J., Thorogood M., \textcolor{magenta}{Tatar K.}, Pasquier P. 2018. Quantitative Analysis of the Impact of Mixing on Perceived Emotion of Soundscape Recordings. In Proceedings of Sound and Music Computing (SMC 2018)}
	
	\cvitem{C6}{**\textcolor{magenta}{Tatar K.}, Pasquier P., \& Siu R. (2018). REVIVE: An Audio-Visual Performance with Musical and Visual Artificial Intelligence Agents. 2018 CHI Conference on Human Factors in Computing Systems, Montreal, QC, Canada ACM 978-1-4503-5621-3/18/04. \url{https://doi.org/10.1145/3170427.3177771}}
	
	\cvitem{C5}{**Prpa M., \textcolor{magenta}{Tatar K.}, Schiphorst T., \& Pasquier P. (2018). Respire: A Breath Away from the Experience in Virtual Environment. 2018 CHI Conference on Human Factors in Computing Systems, Montreal, QC, Canada ACM 978-1-4503-5621-3/18/04. \url{https://doi.org/10.1145/3170427.3180282}}
	
	\cvitem{C4}{ Prpa M., \textcolor{magenta}{Tatar K.}, Fran\c{c}oise J., Riecke B., Schiphorts T., Pasquier P. (2018). Attending to Breath: Exploring How the Cues in a Virtual Environment Guide the Attention to Breath and Shape the Quality of Experience to Support Mindfulness. In Proceedings of the 2018 Designing Interactive Systems Conference (pp. 71-84). ACM Press. \url{https://doi.org/10.1145/3196709.3196765}} 
	
	\cvitem{C3}{\textcolor{magenta}{Tatar, K.} \& Pasquier, P. (2017). MASOM: A Musical Agent Architecture based on Self-Organizing Maps, Affective Computing, and Variable Markov Models. In \textit{Proceedings of the 5th International Workshop on Musical Metacreation} (MuMe 2017).}
	
	\cvitem{C2}{Fan, J., \textcolor{magenta}{Tatar, K.}, Thorogood, M., , \& Pasquier, P. (2017). Ranking Based Experimental Music Emotion Recognition. In \textit{Proceedings of the 18th International Society for Music Information Retrieval Conference} (ISMIR 2017).}
	
	\cvitem{C1}{Prpa, M., \textcolor{magenta}{Tatar, K.}, Riecke, B. E., \& Pasquier, P. (2017). The Pulse Breath Water System: Exploring Breathing as an Embodied Interaction for Enhancing the Affective Potential of Virtual Reality. In \textit{Virtual, Augmented and Mixed Reality, 9th International Conference, VAMR 2017, Held as Part of HCI International 2017, Proceedings}. Vancouver: Springer}
	
	\cvitem{}{\begin{center}\small\textcolor{magenta}{\textcolor{magenta}{Arxiv}}\end{center}}

    \cvitem{AX2}{Dignum V., Casey D., Cerratto-Pargman T., Dignum F., Fantasia V., Formark B., Hammarfelt B., Holmberg G., Holzapfel A., Larsson S., Lagerkvist A., Lakemond N., Lindgren H., Lorig F., Marusic A., Rahm L., Razmetaeva Y., Sikström S., \textcolor{magenta}{Tatar K.}, Tucker J. (Submitted). \textit{On the importance of AI research beyond disciplines.} \textcolor{magenta}{Submitted to \href{https://www.nature.com/ncomms/}{Nature Communications}}. Dignum V. is the first author, and the remaining authors share equal contributions. Preprint: \url{https://arxiv.org/abs/2302.06655}}
 
	\cvitem{AX1}{\textcolor{magenta}{Tatar K.}, Bisig D., \& Pasquier, P. (2020). Introducing Latent Timbre Synthesis. \url{https://arxiv.org/abs/2006.00408}}
	\cvitem{}{\begin{center}\small\textcolor{magenta}{\textcolor{magenta}{Doctoral Thesis}}\end{center}}
	\cvitem{TH1}{Tatar, K. (2019). Musical agents based on self-organizing maps for audio applications [Thesis, Communication, Art \& Technology: School of Interactive Arts and Technology].\url{http://summit.sfu.ca/item/19665}}
	\cvitem{}{}
	\cvitem{}{\small All papers are double-blind peer-reviewed unless otherwise is indicated.}
	\cvitem{}{\small *These papers are single-blind peer-reviewed. The reviewers had access to the list of authors.}
	\cvitem{}{\small **We are invited to submit a paper, and the article is reviewed by the editors.}

\end{document}